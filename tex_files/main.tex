% -----------------------------------
% -----------------------------------
% abnTeX2: Normas ABNT NBR 14724:2011 + sugestões FGV/EMAp. 

% Autor: Lauro César Araujo
% Adaptações EMAp: Lucas Machado Moschen 
% Copyright 2012-2018 by abnTeX2 group at http://www.abntex.net.br/ 

%% This work may be distributed and/or modified under the
%% conditions of the LaTeX Project Public License, either version 1.3
%% of this license or (at your option) any later version.
%% The latest version of this license is in
%%   http://www.latex-project.org/lppl.txt
%% and version 1.3 or later is part of all distributions of LaTeX
%% version 2005/12/01 or later.
% ----------------------------------
% ----------------------------------
\documentclass[
	% -- opções da classe memoir --
	12pt,				% tamanho da fonte
	%openright,			% capítulos começam em página ímpar (insere página vazia caso preciso)
	oneside,			% para impressão em recto e verso. Oposto a oneside
	a4paper,			% tamanho do papel. 
	% -- opções da classe abntex2 --
	%chapter=TITLE,		% títulos de capítulos convertidos em letras maiúsculas
	%section=TITLE,		% títulos de seções convertidos em letras maiúsculas
	%subsection=TITLE,	% títulos de subseções convertidos em letras maiúsculas
	%subsubsection=TITLE,% títulos de subsubseções convertidos em letras maiúsculas
	% -- opções do pacote babel --
	english,			% idioma para inglês
	brazil				% idioma para português
	]{abntex2}

%------------------------------------------------
%-------------- Pacotes necessários -------------
%------------------------------------------------

% Escrita 
\usepackage[T1]{fontenc}
\usepackage[utf8]{inputenc}
\usepackage{lmodern}
\usepackage{microtype} % para melhorias de justificação
\usepackage{indentfirst}

\renewcommand{\ABNTEXchapterfont}{\fontfamily{ptm}\fontseries{b}\selectfont}

% Gráficos 
\usepackage{color}
\usepackage{caption}
\usepackage{subcaption}
\usepackage{graphicx}
\graphicspath{{../../images/}}

% Matemáticos 
\usepackage{amsthm, amssymb, amsmath, mathtools}

% Outros 
\usepackage{lipsum}
\usepackage{makecell}
\usepackage{pbox}
\usepackage{babel}




% Citações 
%\usepackage[brazilian,hyperpageref]{backref}
%\usepackage[alf]{abntex2cite}	% Citações padrão ABNT
\usepackage[style=abnt]{biblatex}
\addbibresource{biblio.bib}  

% \renewcommand{\backrefpagesname}{Citado na(s) página(s):~}
% % Texto padrão antes do número das páginas
% \renewcommand{\backref}{}
% % Define os textos da citação
% \renewcommand*{\backrefalt}[4]{
% 	\ifcase #1 %
% 		Nenhuma citação no texto.%
% 	\or
% 		Citado na página #2.%
% 	\else
% 		Citado #1 vezes nas páginas #2.%
% 	\fi}%
% ---

\input{capa_folha_rosto.tex}

\titulo{Utilization of environmental and epidemiological indicators in the study of malaria dynamics}
\autor{Raphael Felberg Levy}
\local{Rio de Janeiro}
\data{2023}
\instituicao{%
  Fundação Getulio Vargas \\
  \par
  School of Applied Mathematics
}
\tipotrabalho{Bachelor Dissertation (Undergraduation)}

\preambulo{Bachelor dissertation presented to the School of Applied
Mathematics (FGV/EMAp) to obtain the Bachelor's degree in Applied Mathematics. 
\\ \\ Area of Study: Biological modeling.}

\orientador{Flávio Codeço Coelho}

% Se o seu texto tem subtítulo. 
% Se não tiver, altere o arquivo capa_folha_rosto_tex
% \subtitulo{Este é o subtítulo do meu TCC}

%---------------------------------------------
%-------------------- PDF --------------------
%---------------------------------------------

% alterando o aspecto da cor azul
\definecolor{blue}{RGB}{41,5,195}

% informações do PDF
\makeatletter
\hypersetup{
     	%pagebackref=true,
		pdftitle={\@title}, 
		pdfauthor={\@author},
    	pdfsubject={\imprimirpreambulo},
	    pdfcreator={LaTeX with abnTeX2},
		pdfkeywords={abnt}{latex}{abntex}{abntex2}{trabalho acadêmico}, 
		colorlinks=true,       		% false: boxed links; true: colored links
    	linkcolor=blue,          	% color of internal links
    	citecolor=blue,        		% color of links to bibliography
    	filecolor=magenta,      		% color of file links
		urlcolor=blue,
		bookmarksdepth=4
}
\makeatother

% Posiciona figuras e tabelas no topo da página quando adicionadas sozinhas
% em um página em branco. Ver https://github.com/abntex/abntex2/issues/170
\makeatletter
\setlength{\@fptop}{5pt} % Set distance from top of page to first float
\makeatother

%---------------------------------------
%--------- Mais configurações-----------
%---------------------------------------

% Possibilita criação de Quadros e Lista de quadros.
% Ver https://github.com/abntex/abntex2/issues/176
\newcommand{\quadroname}{Quadro}
\newcommand{\listofquadrosname}{Lista de quadros}

\newfloat[chapter]{quadro}{loq}{\quadroname}
\newlistof{listofquadros}{loq}{\listofquadrosname}
\newlistentry{quadro}{loq}{0}

% configurações para atender às regras da ABNT
\setfloatadjustment{quadro}{\centering}
\counterwithout{quadro}{chapter}
\renewcommand{\cftquadroname}{\quadroname\space} 
\renewcommand*{\cftquadroaftersnum}{\hfill--\hfill}

\setfloatlocations{quadro}{hbtp} % Ver https://github.com/abntex/abntex2/issues/176

%-----------------------------------------------------
%--------------------- Margens -----------------------
%-----------------------------------------------------

\setlrmarginsandblock{3cm}{2cm}{*}
\setulmarginsandblock{3cm}{2cm}{*}
\checkandfixthelayout

%-----------------------------------------------------
%------ Espaçamentos entre linhas e parágrafos -------
%-----------------------------------------------------

% O tamanho do parágrafo é dado por:
\setlength{\parindent}{1.3cm}

% Controle do espaçamento entre um parágrafo e outro:
\setlength{\parskip}{0.2cm}  % tente também \onelineskip

% compila o índice
\makeindex

%------------------------------------------------------
%----------- Personal Definitions ---------------------
%------------------------------------------------------

\input{definitions.tex}

%-------------------------------------------------
%----------------- Document ----------------------
%-------------------------------------------------

\begin{document}

\newcounter{num}
% if num != 1, do not print the pre textual 
\setcounter{num}{1}

\selectlanguage{english}
\frenchspacing 

%----------------------------------------------
%--------------- Pré-textuais -----------------
%----------------------------------------------
%\pretextual

\imprimircapa

\ifnum\value{num}=1
{\imprimirfolhaderosto*

%\input{ficha_catalografica.tex}

%\input{errata.tex}

\begin{folhadeaprovacao}

    \begin{center}
      {\ABNTEXchapterfont\large\MakeUppercase{\imprimirautor}}
  
      \vspace*{\fill}\vspace*{\fill}
      \begin{center}
        \ABNTEXchapterfont\bfseries\large\MakeUppercase{\imprimirtitulo}%\normalfont\MakeUppercase
        %{:\imprimirsubtitulo}	
      \end{center}
      \vspace*{\fill}
      
      \hfill
      \begin{minipage}{.7\textwidth}
          \imprimirpreambulo \\ \\
          And approved in 12/12/2023 \\
      \end{minipage}%
      \vspace*{\fill}
     \end{center}
  
     \assinatura{\imprimirorientador \\ School of Applied Mathematics} 
     \assinatura{Claudio José Struchiner \\ School of Applied Mathematics}
     \assinatura{Mônica da Silva-Nunes \\ Universidade Federal de São Carlos - UFSCar}
     %\assinatura{\textbf{Professor} \\ Convidado 3}
     %\assinatura{\textbf{Professor} \\ Convidado 4}
\end{folhadeaprovacao}

% \begin{folhadeaprovacao}
% \includepdf{folhadeaprovacao_final.pdf}
% \end{folhadeaprovacao}

% \begin{dedicatoria}
%     \vspace*{\fill}
%     %\noindent
%     \hfill
%     \begin{minipage}{.6\textwidth}
%      Dedico essa dissertação a todas que lutaram para que eu estivesse aqui. 
%     \end{minipage}
% \end{dedicatoria}
 
\begin{agradecimentos}
    To my family, especially my parents, for all 
    the support and encouragement throughout not 
    only my undergraduate studies but also throughout 
    the entire journey up to this moment.
    \\\\
    To my supervisor, Flávio Codeço Coelho, for being 
    my guide in the development of this work and for 
    introducing me to the field of modeling biological phenomena.
    \\\\
    To all the professors I had the opportunity to 
    meet and from whom I had the pleasure of learning 
    during my undergraduate studies, and to the teaching assistants who were willing to help in the most challenging moments.
    \\\\
    Finally, I would like to express my gratitude to 
    all my friends who accompanied and supported me 
    until now. The last 4 years wouldn't have been the same without you.
\end{agradecimentos}

% \begin{epigrafe}
% \vspace*{\fill}

% \begin{flushright}
%     \hspace{7.5cm}
%     \textit{
%         ``If your experiment needs a statistician, you need a better
%         experiment.''} \\
%         \textit{Ernest Rutherford}
% \end{flushright}
% \end{epigrafe}

\input{resumo.tex}

\pdfbookmark[0]{\listfigurename}{lof}
\listoffigures*
% \begin{itemize}
% 	\item Figura 1: Modelo SIR com dados do artigo de referência
% 	\item Figura 2: Modelo SEI com dados do artigo de referência
% 	\item Figura 3: Gráfico de temperatura anual estimada
% 	\item Figura 4: Gráfico de precipitação anual estimada
% 	\item Figura 5: Modelo SIR com parâmetros adaptados
% 	\item Figura 6: Modelo SEI com parâmetros adaptados
% 	\item Figura 7: Modelo SIR com $k=1$
% 	\item Figura 8: Modelo SEI com $k=1$
% 	\item Figura 9: Gráfico de $\mathcal{R}_0$ em função de $k$
% 	\item Figura 10: Modelo SIR com $k=2.5$
% 	\item Figura 11: Modelo SEI com $k=2.5$
% 	\item Figura 12: Modelo SIR com $k=5$
% 	\item Figura 13: Modelo SEI com $k=5$
% 	\item Figura 14: Modelo SIR com $k=10$
% 	\item Figura 15: Modelo SEI com $k=10$
% 	\item Figura 16: Gráfico de equilíbrio de $I_H$ em função de $k$
% 	\item Figura 17: Gráfico de equilíbrio de $S_H$ em função de $k$
% 	\item Figura 18: Equilíbrio global $S_H^* \times I_H^*$ para $k=10$
% \end{itemize}	
\cleardoublepage

% \pdfbookmark[0]{\listofquadrosname}{loq}
% \listofquadros*
% \cleardoublepage

\pdfbookmark[0]{\listtablename}{lot}
\listoftables*
% \begin{itemize}
% 	\item Tabela 1: Parâmetros usados na modelagem
% 	\item Tabela 2: Parâmetros usados na modelagem
% 	\item Tabela 3: Parâmetros usados na modelagem
% 	\item Tabela 4: População rural de Manaus de 2004 a 2009
% 	\item Tabela 5: Valores dos parâmetros de clima
	
% \end{itemize}	
\cleardoublepage

\input{siglas_simbolos.tex}

}\fi

\pdfbookmark[0]{\contentsname}{toc}
\tableofcontents*
\cleardoublepage

% ----------------------------------------------------------
% ELEMENTOS TEXTUAIS
% ----------------------------------------------------------
\textual

\chapter{Introduction}

The Amazon is one of the largest and most biodiverse tropical forests 
in the world, harboring numerous species of plants, animals, and 
microorganisms, including vectors and pathogens responsible for the 
transmission of various diseases. Among them, one of the most common 
is malaria, caused by protozoa of the genus \textit{Plasmodium}, 
transmitted by the bite of the infected female mosquito of the genus 
\textit{Anopheles}. It is present in 22 American countries, but the 
areas with the highest risk of infection are located in the Amazon 
region, encompassing nine countries, which accounted for $68\%$ of 
infection cases in 
2011 \cite{pimenta_orfano_bahia_duarte_rios-velasquez_melo_pessoa_oliveira_campos_villegas_etal_2015}. 
Although malaria is prevalent in the Americas, it is 
not limited to this continent and is found in countries in Africa and Asia, 
resulting in more than two million cases of infection and 445,000 deaths 
worldwide in 2016 \cite{doi:10.1146/annurev-micro-090817-062712}.
\\\\
Notably, vector-borne disease transmission is closely related to 
environmental changes that interfere with the ecosystem of both 
transmitting organisms and affected organisms. In the case of the 
Amazon, agricultural and livestock settlements are among the factors
that most favor disease transmission, both due to the deforestation 
they cause for establishment and the clustering of people in environments 
close to the vector's habitat \cite{silva-nunes_malaria_amazon_2008}, 
especially by clustering non-immune migrants near these natural and 
artificial breeding sites \cite{DASILVANUNES2012281}.
\\\\
Additionally, other factors such as rainfall, wildfires, and mining 
also significantly influence disease transmission in the region. These 
events result in habitat loss, ecosystem fragmentation, and climate 
changes, affecting the distribution and abundance of vectors and hosts, 
as well as their interaction with pathogens. Furthermore, population growth 
and urbanization also play a crucial role in disease spread, increasing 
human exposure to vectors and infection risks.
\\\\
In this context, this work aims to investigate vector-borne disease 
transmission in the Amazon and analyze how environmental impacts 
influence the dynamics of malaria transmission, the ecological and 
socioeconomic factors affecting this spread, and possible prevention 
and control strategies. The main reference for this research is the 
Trajetórias Project, developed by the Center for Biodiversity and 
Ecosystem Services (SinBiose/CNPq), which is a dataset including 
environmental, epidemiological, economic, and socioeconomic indicators 
for all municipalities in the Legal Amazon, analyzing the spatial and 
temporal relationship between economic trajectories linked to the dynamics 
of agrarian systems, whether they are family-based rural or large-scale 
agricultural and livestock production, the availability of natural resources, 
and the risk of diseases \cite{Rorato2023}.

% ----------------------------------------------------------
% Finaliza a parte no bookmark do PDF
% para que se inicie o bookmark na raiz
% e adiciona espaço de parte no Sumário
% ----------------------------------------------------------
\phantompart

\chapter{Metodology}

For the elaboration of the work, population data from the 
Trajetórias Project dataset and climatic data from the Climate 
Data will be used. Methods of disease transmission based on ordinary 
differential equations, such as the SIR model, will be addressed. 
Starting with a simple modeling approach, environmental phenomena 
such as deforestation and burning will be included to assess how 
modifications in the ecosystem will interfere with the previously 
developed model. Computational calculations were performed using 
the SageMath 9.2 environment, utilizing Scipy's numerical 
integration functions to solve the method.
\\\\
First I will be describing SIR \cite{githubMODBIO, Prasad2022}, 
which can be considered the foundation 
of the models that will be used throughout the project. 
Developed by W. O. Kermack and A. G. McKendrick in 1927, 
SIR is one of the most widely used models for epidemic 
modeling, considering three compartments:

\begin{align*}
    & S: \text{number of susceptible individuals} \\
    & I: \text{number of infected individuals} \\
    & R: \text{number of recovered individuals}
\end{align*}
\\
In this model, healthy individuals in the $S$ class are 
susceptible to contact with individuals in the $I$ class and 
are transferred to this compartment if they contract the disease. 
Infected individuals can spread the disease through 
direct contact with susceptible individuals, but they can 
also become immune over time and are transferred to the $R$ 
compartment. In general, $R$ includes the total of recovered (immune) 
individuals and those who died from the disease, but we can assume 
that the number of deaths is very low compared to the total 
population size and can be ignored. We also assume that individuals 
in this category will not revert to being susceptible or infectious.
\\\\
Considering an epidemic over a short period and 
that the disease is not fatal, we can ignore vital 
dynamics of birth and death. With this, we can describe 
the SIR model through the following system of ordinary 
differential equations (ODEs):

\begin{gather*}
\begin{cases}
\dfrac{dS}{dt} = -\dfrac{\beta SI}{N} \\
\\
\dfrac{dI}{dt} = \dfrac{\beta SI}{N} - \gamma I \\
\\
\dfrac{dR}{dt} = \gamma I
\end{cases}
\end{gather*}
\\
In the model, $N(t) = S(t) + I(t) + R(t)$, i.e., the total population 
at time $t$, while $\beta$ is the infection rate, and $\gamma$ is the
recovery rate. Given that $S + I + R$ is always constant if we 
ignore birth and death, we have 
$\dfrac{dS}{dt} + \dfrac{dI}{dt} + \dfrac{dR}{dt} = 0$.
\\\\
For the disease to spread, it is easy to see that 
$\dfrac{dI}{dt} = \dfrac{\beta SI}{N} - \gamma I > 0$. 
Thus, $\dfrac{\beta SI}{N} > \gamma I \Rightarrow \dfrac{\beta S}{N} > \gamma$. 
Assuming we are at the beginning of the infection, given 
that we want to observe its spread, $I$ will be very small, 
and $S \approx N$. We then conclude that 
$\dfrac{\beta N}{N} > \gamma \Rightarrow \dfrac{\beta}{\gamma} > 1$. 
This dimensionless value can be derived by nondimensionalizing 
the model: 
let $y^* = \dfrac{S}{N}$, $x^* = \dfrac{I}{N}$, $z^* = \dfrac{R}{N}$, 
and $t^* = \dfrac{t}{1/\gamma} = \gamma t$, so that $y^* + x^* + z^* = 1$. 
Substituting the system of ODEs above using these values:


\begin{gather*}
\begin{cases}
\dfrac{dS}{dt} = \dfrac{d(y^*N)}{d(t^*/\gamma)} = -\dfrac{\beta SI}{N} = -\dfrac{\beta(y^*N)(x^*N)}{N} = -\beta y^*Nx^* \\
\\
\dfrac{dI}{dt} = \dfrac{d(x^*N)}{d(t^*/\gamma)} = \dfrac{\beta SI}{N} - \gamma I = \dfrac{\beta(y^*N)(x^*N)}{N} -\gamma(x^*N) = \beta y^*Nx^* - \gamma x^*N \\
\\
\dfrac{dR}{dt} = \dfrac{d(z^*N)}{d(t^*/\gamma)} = \gamma I = \gamma(x^*N)
\end{cases}
\end{gather*}

Now, canceling the factors $N$ and $\gamma$ on both sides of the equations:

\begin{gather*}
\begin{cases}
\dfrac{d(y^*)}{d(t^*)} = -\dfrac{\beta y^*x^*}{\gamma} \\
\\
\dfrac{d(x^*)}{d(t^*)} = \dfrac{\beta y^*x^*}{\gamma} - x^* \\
\\
\dfrac{d(z^*)}{d(t^*)} = x^*
\end{cases}
\end{gather*}
\\
Thus, we have a system given only by $y^*$ and $x^*$ and the 
parameter $\dfrac{\beta}{\gamma}$, which we can call $R_0$.
\\\\
As this work will be primarily focused on malaria modeling, 
I will now present one of the first models developed specifically 
for this disease, by Sir Ronald Ross in 1911 \cite{Bacaër2011}, 
which uses two distinct ordinary differential equations (ODEs) 
different from those presented above:

\begin{gather*}
\begin{cases}
\dfrac{dI}{dt} = bp'i\dfrac{N-I}{N} -aI\\
\\
\dfrac{di}{dt} = bp(n-i)\dfrac{I}{N} - mI
\end{cases}
\end{gather*}
\\
In this case, $N$ is the total human population, $I(t)$ is the number 
of infected humans at time $t$, $n$ is the total mosquito population, 
$i(t)$ is the number of infected mosquitoes at time $t$, $b$ is the 
biting rate, $p$ is the probability of transmission from human to 
mosquito per bite, $p'$ is the probability of transmission from 
mosquito to human per bite, $a$ is the recovery rate of human infection, 
and $m$ is the mosquito mortality rate. $bp'ii\dfrac{N-I}{N}dt -aIdt$ 
represent, respectively, the number of new infected humans and 
the number of recovered humans in the interval $dt$, while 
$bp(n-i)\dfrac{I}{N}dt - mIdt$ represent, respectively, the 
number of new infected mosquitoes and the number of mosquitoes that 
die in that time interval, assuming that infection does not affect 
the mosquito mortality rate.
\\\\
For this model, Ross discussed two equilibrium points, where 
$\dfrac{dI}{dt} = \dfrac{di}{dt} = 0$. They occur when $I=i=0$, 
which is the case where there is no malaria, and, for $I, i > 0$, 
$I = N\dfrac{1-amN/(b^2pp'n)}{1+aN/(bp'n)}$ and 
$i = n\dfrac{1-amN/(b^2pp'n)}{1+m/(bp)}$. Furthermore, for the 
disease to establish itself, $n$ must be greater than a 
threshold value $n^* = \dfrac{amN}{b^2pp'}$. In this case, 
the disease becomes endemic. If $n<n^*$, the equilibrium will 
be at $I=i=0$, and the disease will disappear.
\\\\
Dividing the equations of the equilibrium points by $I \times i$, we have:

\begin{gather*}
\begin{cases}
\dfrac{bp}{N} = \dfrac{bpn}{Ni} -\dfrac{m}{I} \\
\\
\dfrac{bp'}{N} = \dfrac{bp'}{I} -\dfrac{a}{i} 
\end{cases}
\end{gather*}
\\
Which transforms the problem into a linear 
system with two unknowns, $I$ and $i$.
\\\\
Now, I will present the model that will be used 
for the development of the work, based on the one 
developed by Paul E. Parham and Edwin Michael in 
2010, which takes into account factors such as 
rainfall and temperature ($R$ and $T$, respectively) \cite{Parham2010}.
\\\\
Defining the equations that will be used:
\begin{gather*}
\begin{cases}
\dfrac{dS_H}{dt} = -ab_2\bigg(\dfrac{I_M}{N}\bigg)S_H\\
\\
\dfrac{dI_H}{dt} = ab_2\bigg(\dfrac{I_M}{N}\bigg)S_H-\gamma I_H\\
\\
\dfrac{dR_H}{dt} = \gamma I_H\\
\\
\dfrac{dS_M}{dt} = b - ab_1\bigg(\dfrac{I_H}{N}\bigg)S_M - \mu S_M\\
\\
\dfrac{dE_M}{dt} = ab_1\bigg(\dfrac{I_H}{N}\bigg)S_M - \mu E_M - ab_1\bigg(\dfrac{I_H}{N}\bigg)S_Ml(\tau_M)\\
\\
\dfrac{dI_M}{dt} = ab_1\bigg(\dfrac{I_H}{N}\bigg)S_Ml(\tau_M) -\mu I_M
\end{cases}
\end{gather*}
\\\\
It is necessary to mention that the original model 
used $I_M(t-\tau)$ in $\dfrac{dI_H}{dt}$ and $I_H(t-\tau)$ in 
$\dfrac{dE_M}{dt}$ (in the transition from $E$ to $I$) and 
$\dfrac{dI_M}{dt}$, respectively. However, as this would 
make the model based on delay differential equations, it was 
recommended by the advisor to disregard this difference and use 
only the current time $t$.
\\\\
Having the model equations for the human and mosquito 
populations, I will first define the parameters used in the modeling 
and other necessary functions, and then the variables used:
\\
\begin{adjustwidth}{-0.5cm}{}
    \begin{center}
    \renewcommand{\arraystretch}{1.5}
    \raggedleft\begin{tabular}{|c | l | c|} 
     \hline
     \raisebox{-1ex}{\textbf{Parameter}} & \raisebox{-1ex}{\textbf{Definition}} & \raisebox{-1ex}{\textbf{Formula}}\\ 
     \hline
     $T(t)$ & \pbox{8cm}{\rule{0pt}{4.5ex}Temperature\rule[-2.5ex]{0pt}{0pt}} & $T_1 (1 + T_2 \cos(\omega_1t - \phi_1))$\\ 
     \hline
     $R(t)$ & \pbox{8cm}{\rule{0pt}{4.5ex}Precipitation\rule[-2.5ex]{0pt}{0pt}} & $R_1 (1 + R_2 \cos(\omega_2t - \phi_2))$ \\
     \hline
     $b(R, T)$ & \pbox{8cm}{\rule{0pt}{4.5ex}Mosquito birth rate (/ day)\rule[-2.5ex]{0pt}{0pt}} & $\dfrac{B_E  p_E(R)  p_L(R,T)  p_P(R)}{(\tau_E + \tau_L(T) + \tau_P)}$\\ 
     \hline
     $a(T)$ & \pbox{8cm}{\rule{0pt}{4.5ex}Biting rate (/day)\rule[-2.5ex]{0pt}{0pt}} & $\dfrac{(T - T_1)}{D_1}$ \\
     \hline
     $\mu(T)$ & \pbox{8cm}{\rule{0pt}{3ex}Mosquito mortality rate per capita (/ day)\rule[-1.5ex]{0pt}{0pt}} & $-\log(p(T))$ \\
     \hline
     $\tau_M(T)$ & \pbox{8cm}{\rule{0pt}{4.5ex}Duration of the sporozoite cycle (days)\rule[-2.5ex]{0pt}{0pt}} & $\dfrac{DD}{(T - T_{min})}$ \\
     \hline
     $\tau_L(T)$ & \pbox{8cm}{\rule{0pt}{4.5ex}Duration of larval development phase (days)\rule[-2.5ex]{0pt}{0pt}} & $\dfrac{1}{c_1T + c_2}$ \\
     \hline
     $p(T)$ & \pbox{8cm}{\rule{0pt}{3ex}Daily mosquito survival rate\rule[-1.5ex]{0pt}{0pt}} & $e^{(-1 / (AT^2 + BT + C))}$ \\
     \hline
     $p_L(R)$ & \pbox{8cm}{\rule{0pt}{3ex}Probability of larval survival dependent on rainfall\rule[-1.5ex]{0pt}{0pt}} & $(\dfrac{4p_{ML}}{R_L^2})R(R_L - R)$ \\
     \hline
     $p_L(T)$ & \pbox{8cm}{\rule{0pt}{3ex}Probability of larval survival dependent on temperature\rule[-1.5ex]{0pt}{0pt}} & $e^{-(c_1T + c_2)}$ \\
     \hline
     $p_L(R, T)$ & \pbox{8cm}{\rule{0pt}{3ex}Probability of larval survival dependent on temperature and rainfall\rule[-1.5ex]{0pt}{0pt}} & $p_L(R)p_L(T)$ \\
     \hline
     $l(\tau_M)(T)$ & \pbox{8cm}{\rule{0pt}{3ex}Probability of mosquito survival during the sporozoite cycle (/ day)\rule[-1.5ex]{0pt}{0pt}} & $p(T)^{\tau_M(T)}$ \\
     \hline
     $M(t)$ & \pbox{8cm}{\rule{0pt}{3ex}Total number of mosquitoes\rule[-1.5ex]{0pt}{0pt}} & $S_M(t) + E_M(t) + I_M(t)$ \\
     \hline
     $N(t)$ & \pbox{8cm}{\rule{0pt}{3ex}Total number of humans\rule[-1.5ex]{0pt}{0pt}} & $S_H(t) + I_H(t) + R_H(t)$ \\  
     \hline
    \end{tabular}
    \captionof{table}{Parameters used in the modeling}
    \end{center}
    \end{adjustwidth}

    \begin{adjustwidth}{-0.5cm}{}
        \begin{center}
        \renewcommand{\arraystretch}{1.5}
        \raggedleft\begin{tabular}{|c | c|} 
         \hline
         \textbf{Parameter} & \textbf{Definition}\\ 
         \hline
         $b_1$ & \makecell[l]{\rule{0pt}{3ex}Proportion of bites from susceptible mosquitoes \\ on infected humans that result in infection\rule[-1.5ex]{0pt}{0pt}} \\
         \hline
         $b_2$ & \makecell[l]{\rule{0pt}{3ex}Proportion of bites from infected mosquitoes \\ on susceptible humans that result in infection\rule[-1.5ex]{0pt}{0pt}} \\
         \hline
         $\gamma$ & \makecell[l]{\rule{0pt}{3ex}1/Average duration of infectiousness in humans (days$^{-1}$)\rule[-1.5ex]{0pt}{0pt}} \\
         \hline
         $T_1$ & \makecell[l]{\rule{0pt}{3ex}Mean temperature in the absence of seasonality ($^\circ C$)\rule[-1.5ex]{0pt}{0pt}} \\
         \hline
         $T_2$ & \makecell[l]{\rule{0pt}{3ex}Amplitude of seasonal variability in temperature\rule[-1.5ex]{0pt}{0pt}} \\
         \hline
         $R_1$ & \makecell[l]{\rule{0pt}{3ex}Average monthly precipitation in the absence of \\ seasonality (mm)\rule[-1.5ex]{0pt}{0pt}} \\
         \hline
         $R_2$ & \makecell[l]{\rule{0pt}{3ex}Amplitude of seasonal variability in precipitation\rule[-1.5ex]{0pt}{0pt}} \\
         \hline
         $\omega_1$ & \makecell[l]{\rule{0pt}{3ex}Angular frequency of seasonal oscillations in temperature (months$^{-1}$)\rule[-1.5ex]{0pt}{0pt}} \\
         \hline
         $\omega_2$ & \makecell[l]{\rule{0pt}{3ex}Angular frequency of seasonal oscillations in precipitation (months$^{-1}$)\rule[-1.5ex]{0pt}{0pt}} \\
         \hline
         $\phi_1$ & \makecell[l]{\rule{0pt}{3ex}Phase lag of temperature variability (phase shift)\rule[-1.5ex]{0pt}{0pt}} \\
         \hline
         $\phi_2$ & \makecell[l]{\rule{0pt}{3ex}Phase lag of precipitation variability (phase shift)\rule[-1.5ex]{0pt}{0pt}} \\
         \hline
         $B_E$ & \makecell[l]{\rule{0pt}{3ex}Number of eggs laid per adult per oviposition\rule[-1.5ex]{0pt}{0pt}} \\
         \hline
         $p_{ME}$ & \makecell[l]{\rule{0pt}{3ex}Maximum probability of egg survival\rule[-1.5ex]{0pt}{0pt}} \\
         \hline
         $p_{ML}$ & \makecell[l]{\rule{0pt}{3ex}Maximum probability of larval survival\rule[-1.5ex]{0pt}{0pt}} \\
         \hline
         $p_{MP}$ & \makecell[l]{\rule{0pt}{3ex}Maximum probability of pupal survival\rule[-1.5ex]{0pt}{0pt}} \\
         \hline
         $\tau_E$ & \makecell[l]{\rule{0pt}{3ex}Duration of the egg development phase (days)\rule[-1.5ex]{0pt}{0pt}} \\
         \hline
         $b_3^*$ & \makecell[l]{\rule{0pt}{3ex}Infection rate in exposed mosquitoes $(1/\tau_M(T))$\rule[-1.5ex]{0pt}{0pt}} \\
         \hline
        \end{tabular}
        \captionof{table}{Parameters used in the modeling}
        \end{center}
        \end{adjustwidth}


        \begin{adjustwidth}{-0.5cm}{}
            \begin{center}
            \renewcommand{\arraystretch}{1.5}
            \raggedleft\begin{tabular}{|c | c|} 
             \hline
             \textbf{Parameter} & \textbf{Definition}\\ 
             \hline
              $\tau_P$ & \makecell[l]{\rule{0pt}{3ex}Duration of the pupal development phase (days)\rule[-1.5ex]{0pt}{0pt}} \\
             \hline
             $R_L$ & \makecell[l]{\rule{0pt}{3ex}Rainfall threshold until breeding sites are eliminated, \\ removing immature individuals (mm)\rule[-1.5ex]{0pt}{0pt}} \\
             \hline
             $T_{min}$ & \makecell[l]{\rule{0pt}{3ex}Minimum temperature, below which there is no development \\ of the parasite: 14.5 ($^\circ C$)\rule[-1.5ex]{0pt}{0pt}} \\
             \hline
             $DD$ & \makecell[l]{\rule{0pt}{3ex}Degree-days for parasite development. Number of degrees \\ by which the daily average temperature exceeds the minimum \\ development temperature.
             "Sum of heat" for maturation: 105 ($^\circ C \ \text{days}$) \rule[-1.5ex]{0pt}{0pt}} \\
             \hline
             $A$ & \makecell[l]{\rule{0pt}{3ex}Empirical sensitivity parameter ($^\circ C^2 \ \text{days})^{-1}$\rule[-1.5ex]{0pt}{0pt}} \\
             \hline
             $B$ & \makecell[l]{\rule{0pt}{3ex}Empirical sensitivity parameter ($^\circ C \ \text{days})^{-1}$\rule[-1.5ex]{0pt}{0pt}} \\
             \hline
             $C$ & \makecell[l]{\rule{0pt}{3ex}Empirical sensitivity parameter ($\text{days}^{-1}$)\rule[-1.5ex]{0pt}{0pt}} \\
             \hline
             $D_1$ & \makecell[l]{\rule{0pt}{3ex}Constant: 36.5 ($^\circ C \ \text{days}$)\rule[-1.5ex]{0pt}{0pt}} \\
             \hline
             $c_1$ & \makecell[l]{\rule{0pt}{3ex}Empirical sensitivity parameter ($^\circ C \ \text{days})^{-1}$\rule[-1.5ex]{0pt}{0pt}} \\
             \hline
             $c_2$ & \makecell[l]{\rule{0pt}{3ex}Empirical sensitivity parameter ($\text{days}^{-1}$)\rule[-1.5ex]{0pt}{0pt}} \\
             \hline
             $T'^*$ & \makecell[l]{\rule{0pt}{3ex}Empirical temperature parameter ($^\circ C$)\rule[-1.5ex]{0pt}{0pt}} \\
             \hline
            \end{tabular}
            \captionof{table}{Parameters used in the modeling}
            \end{center}
            \end{adjustwidth}

\vspace{1cm}
Parameters marked with $^*$ were added during the development of the modeling to correct 
inaccuracies derived from the original equations in the reference article.
The definition of $DD$ was taken from \cite{McCord2016} and \cite{10665-41724}.
\\\\
Having the equations and parameters, the modeling was initially done using data from the 
rural area of Manaus, in the period from 2004 to 2008, which were selected due to the 
higher incidence of malaria cases caused by $P. \ vivax$, 
the species responsible for the highest number of cases in Brazil 
\cite{OliveiraFerreira2010, 10.3389/fpubh.2021.647754}. Using the incidence 
function used in the Trajetorias project \cite{Rorato2023}, we have:
\begin{gather*}
    \text{Inc}(d, m, z, t_1, t_2) = \dfrac{\text{Cases}(d, m, z, t_1, t_2)}{\text{Pop}(m,z,(t_1+t_2)/2) \times 5 \ \text{years}} \times 10^5,
\end{gather*}
where $\text{Cases}(d, m, z, t_1, t_2)$ is the number of cases of disease $d$ in zone $z$ of municipality 
$m$, and $t_1$ and $t_2$ are the initial and final years of the interval, while 
$\text{Pop}(m,z,(t_1+t_2)/2) \times 5 \ \text{years}$ is the population in zone $z$ 
of municipality $m$ in the middle of the period multiplied by the total number 
of observation years. In this case, we could indicate as:
\begin{gather*}
    \footnotesize{\text{Inc}(\text{Vivax}, \text{Manaus}, \text{Rural}, 2004, 2008) = \dfrac{\text{Cases}(\text{Vivax}, \text{Manaus}, \text{Rural}, 2004, 2008)}{\text{Pop}(\text{Manaus}, \text{Rural}, 2006) \times 5 \ \text{years}} \times 10^5}  \\\\
    184030.8 = \dfrac{78745}{5\text{Pop}} \times 10^5 \Rightarrow Pop \approx 8558
\end{gather*}
Using data on the total population of Manaus in this period, 
with an incidence of 3106.4 and a number of cases of 262264, the 
total population of the municipality was estimated to be 1688540 
inhabitants. Thus, the rural population could be considered as 
approximately 0.5$\%$ of the municipality's population.
\\\\
Having estimated the percentage size of the rural population 
in the city, it was possible to calculate this population for 
each of the years of the analysis through linear interpolation 
using historical series data from IBGE \cite{popIBGE}:
\\\\
\begin{adjustwidth}{0cm}{}
\begin{center}
\renewcommand{\arraystretch}{1.5}
\begin{tabular}{|c | c|} 
 \hline
 \textbf{Year} & \textbf{Estimated rural population}\\ 
 \hline
$2004$ & $7717$ \\
 \hline
 $2005$ & $7889$ \\
 \hline
 $2006$ & $8061$ \\
 \hline
 $2007$ & $8233$ \\
 \hline
 $2008$ & $8492$ \\
 \hline
 $2009$ & $8751$ \\
 \hline
\end{tabular}
\captionof{table}{Manaus' rural population from 2004 to 2009}
\end{center}
\end{adjustwidth}

\vspace{1cm}
As there were population data for the years 2000, 2007, and 2010, 
interpolations were performed with different initial and final 
points, using data from 2000 to 2007 for 2004-2007 and from 2007 
to 2010 for 2008-2009, ensuring the correct use of the 2007 population.
\\\\
Now, describing a bit of the theory behind environmental factors, 
according to \cite{Norris2004}, the removal of tree canopies allowed 
the resurgence of malaria in South America. In deforested areas, 
without tree canopies covering the ground, water puddles under sunlight 
attract mosquitoes of the species $Anopheles \ darlingi$, the main vector 
related to human malaria in the Amazon \cite{infoAnopheles}. They are 
usually less commonly found in still intact forests. This is 
because light and heat favor the development of larvae and 
pupae, in addition to a greater availability of algae for 
larval feeding \cite{article_alteracoesambientais}. The increase 
in ambient temperature also favors the vectorial capacity of 
mosquitoes. Deforestation also attracts and brings humans closer 
to take part in logging, agriculture, and road construction 
activities, bringing individuals infected with $Plasmodium$ to 
an area where both the vector and the environment have already 
been modified to favor transmission. Furthermore, agriculture 
also promotes river sedimentation, providing suitable environments 
for breeding sites. Therefore, it can be considered a relevant 
change for the model to take into account deforestation, the 
increase in survival probabilities of eggs, larvae, and pupae, 
as well as increasing the proportion of bites that lead to infection, 
due to the increased human population density in areas near mosquito 
breeding sites.


\chapter{Results}


Analyzing the results obtained with the original parameters from 
the article by Parham and Michael 
($T' = 19.9 \ (\text{also called $T_1$ in the article}), \ T_1=23.2, \ T_2=0.07, \ \omega_1=0.67, \ \phi_1=1.53, \ R_1=85.9, \ R_2=0.98, 
\ \omega_2=0.65, \ \phi_2=1.99, \ A=-0.03, \ B=1.31, \ C=-4.4, \ b_1=0.04, \ 
b_2 = 0.09, \ T_{min}=14.5, \ \gamma= 1/120, \ R_L = 50, \ c_1=0.00554, \ 
c_2=-0.06737$ \cite{Parham2010}, \cite{OKUNEYE201772}), and using 
the previously estimated average population and an arbitrary value 
for the mosquito population, of 10000, assuming 1000 infected humans and 
5000 exposed mosquitoes at $t=0$, the modeling is as follows
\footnote{The development of the model with the original data can be found at 
https://github.com/RaphaLevy/Undergraduate\_Dissertation/blob/main/
\\modeling\_files/Original\_Parameters.ipynb.}: 

% Pra plotar duas imagens uma ao lado da outra, precisa corrigir a posição das legendas.

% \begin{figure}
% \hspace*{-1.5cm} % Adiciona espaço negativo para puxar a imagem para a esquerda
% \begin{minipage}{.45\textwidth}
%   \centering
%   \includegraphics[width=1.25\linewidth]{SIR_Dados_Originais_Parham_Michael.png}
%   \captionof{figure}{A figure}
%   \label{fig:test1}
% \end{minipage}%
% \hspace{1.5cm} % Adiciona espaço horizontal
% \begin{minipage}{.45\textwidth}
%   \centering
%   \includegraphics[width=1.3\linewidth]{SEI_Dados_Originais_Parham_Michael.png}
%   \captionof{figure}{Another figure} % Legenda à direita da segunda imagem
%   \label{fig:test2}
% \end{minipage}
% \end{figure}





\begin{figure}[!ht]
        \centering
        \hbox{\hspace{6em} \includegraphics[scale=0.45] {THESIS-SIR_Dados_Originais_Parham_Michael_CORRECAO.png}}
        \caption{SIR with original parameters}
\end{figure} 
\begin{figure}[!ht]
        \centering
        \hbox{\hspace{6em} \includegraphics[scale=0.45] {THESIS-SEI_Dados_Originais_Parham_Michael_CORRECAO.png}}
        \caption{SEI with original parameters}
\end{figure} 
\newpage
With this initial modeling, 
a strong oscillation in the number 
of susceptible humans and mosquitoes, 
as well as infected humans, is noticeable. 
Furthermore, it is evident that with these 
parameters, the disease will not become endemic, 
as the number of infected humans tends to 0 
throughout the year, while the population of 
susceptible mosquitoes becomes negative, and the 
population of exposed and infected individuals also tends to 0. 
These effects were characterized by temperature and 
precipitation oscillating in very short periods of 
time, attributed to a high value of $\omega$ for both functions.
\\\\
Starting with only a single infected human and exposed mosquito, the 
oscilatting population isn't noticeable in the SIR plot, however the mosquito population 
still becomes negative.
\\\\
Now, the first necessary modification is to correct the temperature 
and precipitation to consider data from Manaus, as the original 
paper uses data from Tanzania. So, 
collecting climatological data from Manaus from \cite{ClimaMANAUS}, 
the average temperature and precipitation were estimated 
as 26.4 $^\circ C$ and 250.083 mm, respectively. 
With this data, the amplitude of seasonal variability, 
angular frequency, and phase lag of variability for 
both were defined to approximate the real values:
\\\\
\begin{adjustwidth}{-0.5cm}{}
\begin{center}
\renewcommand{\arraystretch}{1.5}
\begin{tabular}{|c | c|} 
 \hline
 \textbf{Parameter} & \textbf{Value}\\ 
 \hline
  $T_1$ & \makecell[l]{\rule{0pt}{3ex}26.4$^\circ C$\rule[-1.5ex]{0pt}{0pt}} \\
 \hline
 $T_2$ & \makecell[l]{\rule{0pt}{3ex}0.025\rule[-1.5ex]{0pt}{0pt}} \\
 \hline
 $\omega_1$ & \makecell[l]{\rule{0pt}{3ex}0.017 (months)$^{-1}$\rule[-1.5ex]{0pt}{0pt}} \\
 \hline
 $\phi_1$ & \makecell[l]{\rule{0pt}{3ex}-1.45\rule[-1.5ex]{0pt}{0pt}} \\
 \hline
 $R_1$ & \makecell[l]{\rule{0pt}{3ex}250.083 mm\rule[-1.5ex]{0pt}{0pt}} \\
 \hline
 $R_2$ & \makecell[l]{\rule{0pt}{3ex}0.565\rule[-1.5ex]{0pt}{0pt}} \\
 \hline
 $\omega_2$ & \makecell[l]{\rule{0pt}{3ex}0.02 (months)$^{-1}$\rule[-1.5ex]{0pt}{0pt}} \\
 \hline
 $\phi_2$ & \makecell[l]{\rule{0pt}{3ex}1.6\rule[-1.5ex]{0pt}{0pt}} \\
 \hline
\end{tabular}
\captionof{table}{Values for climatic parameters}
\end{center}
\end{adjustwidth}

\vspace{1cm}
The amplitude parameters ($T_2$ and $R_2$) and phase lag 
parameters ($\phi_1$ and $\phi_2$) are dimensionless. 
The temperature and precipitation throughout the year 
then evolve as follows
\footnote{The development of the model with the original data can be found at 
https://github.com/RaphaLevy/Undergraduate\_Dissertation/blob/main/
\\modeling\_files/Adapting\_T\_and\_R.ipynb.}:

% \begin{figure}
% \hspace*{-1.5cm} % Adiciona espaço negativo para puxar a imagem para a esquerda
% \begin{minipage}{.45\textwidth}
%   \centering
%   \includegraphics[width=1.2\linewidth]{Grafico_da_Temperatura.png}
%   \captionof{figure}{A figure}
%   \label{fig:test1}
% \end{minipage}%
% \hspace{1.5cm} % Adiciona espaço horizontal
% \begin{minipage}{.45\textwidth}
%   \centering
%   \includegraphics[width=1.2\linewidth]{Grafico_da_Precipitacao.png}
%   \captionof{figure}{Another figure} % Legenda à direita da segunda imagem
%   \label{fig:test2}
% \end{minipage}
% \end{figure}


\begin{figure}[!ht]
        \centering
        \hbox{\hspace{6.0em} \includegraphics[scale=0.7] {THESIS-Grafico_da_Temperatura.png}}
        \caption{Temperature graph}
\end{figure} 
\begin{figure}[!ht]
        \centering
        \hbox{\hspace{6.5em} \includegraphics[scale=0.7] {THESIS-Grafico_da_Precipitacao.png}}
        \caption{Precipitation graph}
\end{figure} 
In order to ensure the correctness of the function 
with the parameters used, I calculated the temperature 
values for the months of October and May, which are the 
hottest and coldest months of the year with average 
temperatures of 27.6 $^\circ C$ and 25.8 $^\circ C$, respectively. 
Additionally, I calculated the precipitation values in March 
and August, which are the months with the highest and lowest 
precipitation, with 395 mm and 114 mm, respectively. 
The obtained average values were 27.06 $^\circ C$, 25.86 $^\circ C$, 
390.67 mm, and 112.89 mm. 
\\\\With the temperature and precipitation parameters ready for an 
initial analysis, the evolution of human and mosquito populations 
was verified.  
The results can be found in Appendices 1 and 2 
\footnote{The development of the graphs above can be found at 
https://github.com/RaphaLevy/Undergraduate\_Dissertation/blob/main/
\\modeling\_files/Adapting\_T\_and\_R.ipynb.}. 
\\\\
Notably, this result is still incorrect, as the mosquito population
still becomes negative over time. 
However, the daily oscillations
was indeed eliminated, due to the modifications made to $\omega_1$ and $\omega_2$.
Another parameter that should also be adapated in consideration for our new climatic
values was $R_L$, the rainfall limit beyond which breeding sites get flushed out and no 
immature stages survive. In the original paper, $R_L$ was defined as 50 mm, which is
too low for the high precipitation in the Amazon, which has a maximum of almost
400 mm in March. Because of this, I increased $R_L$ from 50 mm to 450 mm.
The results can be found in Appendices 3 and 4 
\footnote{The development of the graphs above can be found at 
https://github.com/RaphaLevy/Undergraduate\_Dissertation/blob/main/
\\modeling\_files/Adapting\_T\_and\_R.ipynb.}.
\\\\
Now, while the mosquito population in non-negative, it rapidly declines and
becomes extinct. That is due to the value of $\mu$, the mortality rate of mosquitoes,
given by $-\log(e^{(-1 / (AT^2 + BT + C))})$. For small values of 
$A$, $B$ and $C$, mostly $A$, $\mu$ becomes ``relatively'' large, and the mosquito
population goes extinct. To fix this, I altered these 3 parameters from the initial values 
of the arcticle, which had $\mu \approx 0.0027$, to $A=356.3, \ B=15$
and $C=-48.78$, which gave a $\mu$ of approximately $4.02133 \times 10^{-6}$, 
and so the mosquto population is nearly constant throughout the analysis.
The results, with $A, \ B$
and $C$ as above, can be found in Appendices 5 and 6 
\footnote{The development of the graphs above can be found at 
https://github.com/RaphaLevy/Undergraduate\_Dissertation/blob/main/
\\modeling\_files/Modifying\_Constant\_Parameters.ipynb.}.
\\\\
Now, it is possible to see that the mosquito population in fact remains nearly constant,
as there is close to no transference between compartments.
\\\\
To verify what else could be done to allow the 
transference of individuals between compartments,
given the differential equations of the SEI model and the 
parameters provided to achieve this goal, the value of $\mu$ becomes 
very close to 0, as seen above, while $l(\tau_M)$, a probability, becomes very 
close to 1. Therefore, $\dfrac{dE_M}{dt}$ also becomes very close to 0, 
causing the exposed function to be linear, while the mosquito population 
leaving the susceptible compartment almost simultaneously enters 
the infected compartment, causing mirrored oscillations of $S$ and $I$. 
To overcome this effect, it was necessary to modify the use of $b_1$ to 
only move mosquitoes from compartment $S$ to $E$, requiring the inclusion 
of a new parameter, $b_3$, to move mosquitoes from compartment $E$ to $I$. 
This rate is inversely related to the incubation period, so we define
\begin{gather*}
b_3 = \dfrac{T-T_{min}}{DD}
\end{gather*}
Moreover, the parameter $a(T)$ in the transition from 
exposed mosquitoes to infected was removed, as no new bites occur 
in this change from compartment $E$ to $I$.
\\\\
With these modifications, the differential equations of the SEI model were modified:
\begin{gather*}
\begin{cases}
\dfrac{dS_M}{dt} = b - ab_1\bigg(\dfrac{I_H}{N}\bigg)S_M - \mu S_M\\
\\
\dfrac{dE_M}{dt} = ab_1\bigg(\dfrac{I_H}{N}\bigg)S_M - \mu E_M - b_3E_Ml\\
\\
\dfrac{dI_M}{dt} = b_3E_Ml -\mu I_M\\
\end{cases}
\end{gather*}
\\\\
The results can be seen in Apendices 7 to 10, using $T'=19.9^\circ C$, as originally used,
and $T'=24.4^\circ C$, which is a value lower than the minimum temperature, close to $25.6^\circ C$.
Using $T'=19.9^\circ C$, the human population remains close to what could be seen in Appendix 5,
while the mosquito population is now oscillating, with the number of exposed 
mosquitoes decreasing to values near 0, while the infected population increases.
With $T'=24.4^\circ C$, the SEI plot is very similar, however the increase 
of the infected population is less noticeable.
In the case of the human population, it can be seen that the disease begins 
at a later time, and the maximum number of infected individuals is lower than
what was seen before
\footnote{The development of the graphs above can be found at 
https://github.com/RaphaLevy/Undergraduate\_Dissertation/blob/main/
\\modeling\_files/SEI\_Exposed\_to\_Infected.ipynb.}.
\\\\
Another necessary modification was to disassociate $l$, the probability of mosquito 
survival during the sporozoite cycle, from the infection rate of exposed individuals $b_3$.
\begin{gather*}
        \begin{cases}
        \dfrac{dS_M}{dt} = b - ab_1\bigg(\dfrac{I_H}{N}\bigg)S_M - \mu S_M\\
        \\
        \dfrac{dE_M}{dt} = ab_1\bigg(\dfrac{I_H}{N}\bigg)S_M - \mu E_M - b_3E_M -lE_M\\
        \\
        \dfrac{dI_M}{dt} = b_3E_M -\mu I_M\\
        \end{cases}
        \end{gather*}
With this alteration, it was noted that both populations didn't reach an equilibrium
in the first year of analysis, so extending it to 5 years, the results were as seen in Appendices 11 to 14 
\footnote{The development of the graphs above can be found at 
https://github.com/RaphaLevy/Undergraduate\_Dissertation/blob/main/
\\modeling\_files/SEI\_Exposed\_to\_Infected.ipynb.}. 
\\\\
With the SIR/SEI model now properly corrected, it was possible 
to proceed with the analyses of deforestation application. 
To do so, I began by calculating the $\mathcal{R}_0$ for both SIR and SEI, 
as well as for the coupled model, using the formulation from 
P. van den Driessche \cite{VANDENDRIESSCHE200229} as a reference:
\\\\
Firstly, we define $X_s$ as the set of all disease-free states, 
$$X_s=\{x \geq 0|x_i=0, i=1,\ldots,m\},$$
where $X=(x_1,\ldots, x_n)^T$, such that $x_i\geq 0$ represents the number 
of individuals in each compartment, and we assume each function to be 
continuously differentiable at least twice in each variable ($C^2$).
\\\\
Now, we rearrange the equations so that the first $m$ equations 
contain the infected individuals. Let ${\mathcal F}_i(x)$ be the 
rate of appearance of new infections in compartment $i$, 
${\mathcal V}_i^+(x)$ be the rate of individuals entering 
compartment $i$ by other means, and ${\mathcal V}_i^-(x)$ 
be the rate of individuals leaving compartment $i$. The disease 
transmission model consists of non-negative initial conditions 
together with the following system of equations:
$$\dot{x}=f_i(x)={\mathcal F}_i(x)-{\mathcal V}_i(x), i=1,\ldots, n,$$
where ${\mathcal V}_i (x) = {\mathcal V}_i^{-}(x) - {\mathcal V}_i^+(x)$. 
We also define 
$F=\left[\frac{\partial {\mathcal F}_i (x_0)}{\partial x_j}\right]$ 
and $V=\left[\frac{\partial {\mathcal V}_i (x_0) }{\partial x_j}\right]$, 
where $x_0$ is a Disease-Free Equilibrium (DFE), and $1\leq i,j \leq m$.
\\\\
This is equivalent to the Jacobian of these two matrices, 
after substituting $x_0$, i.e., $S=1$. $\mathcal{R}_0$ will be given 
by $\rho(FV^{-1})$, in other words, it will be the spectral radius 
of the matrix $FV^{-1}$. With the necessary definitions, we can 
calculate the $\mathcal{R}_0$ for both models as follows:
\begin{itemize}
\item \textbf{SIR:}
In this case, $m=1$, and our compartments will be arranged as $[I_H, S_H, R_H]$. Since $\mathcal{R}_0$ is calculated with normalized values, we will multiply the necessary equations by $N$ to remove the denominator. Specifically, for the SIR case, as $R_H$ is not used in any of the equations, we can express it solely in terms of $S$ and $I$. Therefore:
$$ {\mathcal F}_i(x): \text{ rate of appearance of new infected individuals in compartment } i $$
$$ {\mathcal F} =\begin{bmatrix}
a  b_2  I_M  S_H \\
\end{bmatrix} $$
Additionally, we have
$$ {\mathcal V}_i(x)^-: \text{ rate of leaving compartment } i $$
$$ {\mathcal V}_i(x)^+: \text{ rate of entering compartment } i $$
Thus:
$$
{\mathcal V^-} = \begin{bmatrix}
\gamma I_H\\
\end{bmatrix}
$$
$$
{\mathcal V^+} = \begin{bmatrix}
0\\
\end{bmatrix}
$$
$${\mathcal V}_i (x) = {\mathcal V}_i(x)^{-} - {\mathcal V}_i(x)^+$$
Therefore,
$$
{\mathcal V} =
\begin{bmatrix}
\gamma I_H\\
\end{bmatrix}
$$
\\
Hence
$$ F = \dfrac{\partial{\mathcal F}}{\partial I_M} =\begin{bmatrix}
a  b_2  S_H \\
\end{bmatrix} $$
$$ V = \dfrac{\partial{\mathcal V}}{\partial I_H} =\begin{bmatrix}
\gamma \\
\end{bmatrix} $$
\\At the equilibrium, $[S_H^*, I_H^*] = [1,0]$, so $F=[a  b_2], \ V = [\gamma]$ and $\mathcal{R}_0 = \Big | \dfrac{ab_2}{\gamma}\Big | $.

\item \textbf{SEI:}
In this case, $m=2$, and our compartments will be arranged as $[E_M, I_M, S_M]$. Again, we multiply the necessary equations by $N$ to remove the denominator. Therefore:
$$ {\mathcal F} =\begin{bmatrix}
a b_1 I_H S_M\\
0\\
\end{bmatrix} $$
$$
{\mathcal V^-} = \begin{bmatrix}
E_M (\mu + b_3 + l)\\
\mu I_M
\end{bmatrix}
$$
$$
{\mathcal V^+} = \begin{bmatrix}
0\\
b_3 E_M\\
\end{bmatrix}
$$
$${\mathcal V}_i (x) = {\mathcal V}_i(x)^{-} - {\mathcal V}_i(x)^+$$
Thus,
$$
{\mathcal V} =
\begin{bmatrix}
E_M (\mu + b_3 + l)\\
\mu I_M - b_3 E_M\\
\end{bmatrix}
$$
\\
Hence
$$ F = \dfrac{\partial{\mathcal F}}{\partial E_M, I_H} =\begin{bmatrix}
\dfrac{\partial ab_1 I_H S_M}{\partial E_M} & \dfrac{\partial ab_1 I_H S_M}{\partial I_H}\\
\dfrac{\partial 0}{\partial E_M} & \dfrac{\partial 0}{\partial I_H}\\
\end{bmatrix} = 
\begin{bmatrix}
0 & ab_1 S_M\\
0 & 0\\
\end{bmatrix}$$
$$ V = \dfrac{\partial{\mathcal V}}{\partial E_M, I_M} =\begin{bmatrix}
\dfrac{\partial E_M (\mu + b_3 + l)}{\partial E_M} & \dfrac{\partial E_M (\mu + b_3 + l)}{\partial I_M}\\
\dfrac{\partial \mu I_M - b_3 E_M}{\partial E_M} & \dfrac{\partial \mu I_M - b_3 E_M}{\partial I_M}\\
\end{bmatrix} = 
\begin{bmatrix}
\mu+b_3+l & 0\\
- b_3 & \mu\\
\end{bmatrix}$$
\\At the equilibrium, $[S_M^*, E_M^*, I_M^*] = [1,0,0]$, so $$F=\begin{bmatrix}
0 & ab_1\\
0 & 0\\
\end{bmatrix},$$
$$V = \begin{bmatrix}
\mu+b_3+l & 0\\
- b_3 & \mu\\
\end{bmatrix}$$ and $\mathcal{R}_0 = \Big | \dfrac{ab_1b_3}{(b_3+l+\mu)\mu}\Big | $.

\item \textbf{SIR/SEI:}
In this case, $m=3$, and our compartments will be arranged as $[I_H, E_M, I_M, S_H, S_M]$. Once again, we multiply the necessary equations by $N$ to remove the denominator. Therefore:
$$ {\mathcal F} =\begin{bmatrix}
a  b_2  I_M  S_H \\
a b_1 I_H S_M\\
0\\
\end{bmatrix} $$
$$
{\mathcal V^-} = \begin{bmatrix}
\gamma I_H\\
E_M (\mu + b_3 + l)\\
\mu I_M
\end{bmatrix}
$$
$$
{\mathcal V^+} = \begin{bmatrix}
0\\
0\\
b_3 E_M \\
\end{bmatrix}
$$
$${\mathcal V}_i (x) = {\mathcal V}_i(x)^{-} - {\mathcal V}_i(x)^+$$
Thus,
$$
{\mathcal V} =
\begin{bmatrix}
I_H \gamma \\
E_M (\mu + b_3 + l)\\
\mu I_M - b_3 E_M\\
\end{bmatrix}
$$
\\
Hence
$$ F = \dfrac{\partial{\mathcal F}}{\partial I_H, E_M, I_M} =\begin{bmatrix}
\dfrac{\partial ab_2 I_M S_H}{\partial I_H} & \dfrac{\partial ab_2 I_M S_H}{\partial E_M} & \dfrac{\partial ab_2 I_M S_H}{\partial I_M}\\
\dfrac{\partial ab_1 I_H S_M}{\partial I_H} & \dfrac{\partial ab_1 I_H S_M}{\partial E_M} & \dfrac{\partial ab_1 I_H S_M}{\partial I_M}\\
\dfrac{\partial 0}{\partial I_H} & \dfrac{\partial 0}{\partial E_M} & \dfrac{\partial 0}{\partial I_M}\\
\end{bmatrix} = 
\begin{bmatrix}
0 & 0 & ab_2 S_H\\
ab_1 S_M & 0 & 0\\
0 & 0 & 0
\end{bmatrix}$$
$$ V = \dfrac{\partial{\mathcal V}}{\partial I_H, E_M, I_M} =\begin{bmatrix}
\dfrac{\partial \gamma I_H}{\partial I_H} & \dfrac{\partial \gamma I_H}{\partial E_M} & \dfrac{\partial \gamma I_H}{\partial I_M}\\
\dfrac{\partial E_M (\mu + b_3 + l)}{\partial I_H} & \dfrac{\partial E_M (\mu + b_3 + l)}{\partial E_M} & \dfrac{\partial E_M (\mu + b_3 + l)}{\partial I_M}\\
\dfrac{\partial \mu I_M - b_3 E_M}{\partial I_H} & \dfrac{\partial \mu I_M - b_3 E_M}{\partial E_M} & \dfrac{\partial \mu I_M - b_3 E_M}{\partial I_M}\\
\end{bmatrix} = 
$$
$$
\begin{bmatrix}
\gamma & 0 & 0\\
0 & b_3+l+\mu & 0\\
0 & -b_3 & \mu
\end{bmatrix}$$
\\At the equilibrium, $[S_H^*, S_M^*, I_H^*, E_M^*, I_M^*] = [1,1,0,0,0]$, so $$F=\begin{bmatrix}
0 & 0 & ab_2\\
ab_1 & 0 & 0\\
0 & 0 & 0
\end{bmatrix},$$
$$V = \begin{bmatrix}
\gamma & 0 & 0\\
0 & b_3+l+\mu & 0\\
0 & -b_3 & \mu
\end{bmatrix}$$ 
and $\mathcal{R}_0 = \Big | \sqrt{\dfrac{a^2 b_1 b_2 b_3}{(b_3 + l + \mu)\gamma \mu}}\Big | = 
\sqrt{\mathcal{R}_{0 SIR} \times \mathcal{R}_{0 SEI}}$. 
\end{itemize}
We can verify that $\mathcal{R}_0$ is indeed dimensionless: 
$a, \ b_3, \ l, \ \mu$, and $\gamma$ are functions with units 
of 1/day, while $b_1$ and $b_2$ are dimensionless. Therefore, 
$\mathcal{R}_0$ for the SIR model has dimensions of (1/day)/(1/day), 
$\mathcal{R}_0$ for the SEI model has dimensions of (1/day$^2$)/(1/day$^2$), 
and the coupled model has dimensions of (1/day$^3$)/(1/day$^3$).
\\\\
Before continuing with the modeling, it was decided to analyze the evolution
of the rates as a function of temperature and precipitation, instead of time,
to see their behavior as $T$ and $R$ vary. In this case, $T'$ was used 
with value $25.6^{\circ}C$, in order to approximate
$\mathcal{R}_0$ to 0.5, which will be used later in the dissertation. 
The results can be found starting on
Appendix 15 
\footnote{The development of the graphs above can be found at 
https://github.com/RaphaLevy/Undergraduate\_Dissertation/blob/main/
\\modeling\_files/Plotting\_Rates.ipynb.}.
\\\\
Now, having the formula for $\mathcal{R}_0$ for the coupled model,
the model was executed starting with $E_{M0}=1$ and $I_{M0}=1$ and with
$E_{M0}=50000$ and $I_{M0}=1000$
\footnote{These results can be found respectively at
https://github.com/RaphaLevy/Undergraduate\_Dissertation/blob/main/
\\modeling\_files/Plotting\_with\_R0.ipynb and
https://github.com/RaphaLevy/Undergraduate\_Dissertation/blob/main/
\\modeling\_files/Plotting\_with\_R0\_2.ipynb.}. With the parameters used
in Appendices 13 and 14, $\mathcal{R}_0$ was calculated to be 5.99, which
means the disease becomes endemic, as it's value is greater than 1. Using $T'=25.6^{\circ}C$
and $A=12.5$, the results were seen as below:
\newpage
\begin{figure}[!ht]
        \centering
        \hbox{\hspace{2.8em} \includegraphics[scale=0.5] {THESIS-SIR_Initial_EI_1_R0_leq_1.png}}
        \caption{SIR with $\mathcal{R}_0 <1$, starting with $E_{M0}$ and $I_{M0}=1$}
\end{figure} 
\begin{figure}[!ht]
        \centering
        \hbox{\hspace{2.8em} \includegraphics[scale=0.5] {THESIS-SEI_Initial_EI_1_R0_leq_1.png}}
        \caption{SEI with $\mathcal{R}_0 <1$, starting with $E_{M0}$ and $I_{M0}=1$}
\end{figure}
\newpage
\begin{figure}[!ht]
        \centering
        \hbox{\hspace{2.5em} \includegraphics[scale=0.5] {THESIS-SIR_Initial_EI_geq_1_R0_leq_1.png}}
        \caption{SIR with $\mathcal{R}_0 <1$, starting with $E_{M0}$ and $I_{M0}>1$}
\end{figure} 
\begin{figure}[!ht]
        \centering
        \hbox{\hspace{2.5em} \includegraphics[scale=0.5] {THESIS-SEI_Initial_EI_geq_1_R0_leq_1.png}}
        \caption{SEI with $\mathcal{R}_0 <1$, starting with $E_{M0}$ and $I_{M0}>1$}
\end{figure}
With this, as expected, we can see that the disease dies out, specifically 
the susceptible mosquito population. This would be due to the mortality rate
being higher than the birth rate.
\\\\Having the model ready, it was possible to proceed with the analysis of
the impacts of deforestation. To do that, it was decided to include a multypling
factor $k$ for the infection rates of both models, substituting $ab_1$ and
$ab_2$ by $kab_1$ and $kab_2$, respectively. This factor will be used to represent 
the increasing contact between human and mosquito populations due to the
the approximation caused by this environental change. With that in mind, 
we are interested to see how the disease behaves as $k$ increases, therefore
as the contact between the populations increases.
\\\\
To do this, it was also decided that, instead of using the estimated population 
of the rural region of Manaus of 8558, use the population estimated via interpolation 
for 2004, and include a birth and mortality rate for the human population, $\mu_H$. 
With the previously obtained values for the human population 
between 2004 and 2009, the annual birth rate was 
estimated to be 206.8 births per year. Therefore, this corresponds 
to approximately 0.56657 births per day and 0.00007 births per day 
per person, given that the average rural population in Manaus is 
approximately 8078.5 people between 2004 and 2008. Therefore, using $\mu_h = 0.00007$ 
and $k$, the final model is given by:
\begin{gather*}
        \begin{cases}
        \dfrac{dS_H}{dt} = \mu_HN-akb_2\bigg(\dfrac{I_M}{N}\bigg)S_H - \mu_HS_H\\
        \\
        \dfrac{dI_H}{dt} = akb_2\bigg(\dfrac{I_M}{N}\bigg)S_H-\gamma I_H - \mu_HI_H\\
        \\
        \dfrac{dR_H}{dt} = \gamma I_H - \mu_HR_H\\
        \\
        \dfrac{dS_M}{dt} = b - akb_1\bigg(\dfrac{I_H}{N}\bigg)S_M - \mu S_M\\
        \\
        \dfrac{dE_M}{dt} = akb_1\bigg(\dfrac{I_H}{N}\bigg)S_M - \mu E_M - b_3E_M -lE_M\\
        \\
        \dfrac{dI_M}{dt} = b_3E_M -\mu I_M\\
        \end{cases}
        \end{gather*}
Where $\mathcal{R}_0$ for SIR is given by $\Big | \dfrac{akb_2}{\gamma + \mu_H}\Big | $,
for SEI is given by $\Big | \dfrac{akb_1b_3}{(b_3 + l + \mu)\mu}\Big | $
and for the coupled model is given by $\Big | \sqrt{\dfrac{a^2k^2b_1b_2b_3}{(b_3+l+\mu)(\gamma+\mu_H)\mu}}\Big | $.
It can be immediately seen that $k$ will have a linear impact on
$\mathcal{R}_0$.
\\\\
The results using $k=1.5, \ 2, \ 2.5, \ 5$ and $10$ can be found in Appendices 
27 to 36, while plots for each population for multiple values 
of $k$ can be found in Appendices 37 to 42. The results with $k=1$ 
is seen in Figures 5 and 6 \footnote{These results can be found respectively at
https://github.com/RaphaLevy/Undergraduate\_Dissertation/blob/main/
\\modeling\_files/Plotting\_with\_k.ipynb.}.
\\\\
It can be seen that, even starting with a single infected 
human and exposed mosquito,
the disease will eventually become endemic. For $k$ ranging between 1 and 2,
$\mathcal{R}_0$ is less than 1, so the disease will die out. However, for $k=2.5$, 
$\mathcal{R}_0>1$, and it is possible to see, even if only at the end of the analysis, that the disease 
starts to become endemic, as the human population of infected and recovered 
have a slight increase by $t=1500$. For the mosquito population, this 
behavior is less evident.
\\\\
For $k=5$ and $k=10$, the disease is already endemic, as it is possible to see
how it evolves, and that the population will eventually be nearly completely 
recovered from malaria in the case of the human population. Comparing the results 
of SIR between $k=5$ and $k=10$, the most notable difference is that, the larger 
the value of $k$, consequently the larger the value of $\mathcal{R}_0$, 
the faster the disease becomes endemic. For the SEI model, the difference 
is that the population stabilizes at different ranges. For $k$ up to 2.5, 
it appears that the disease doesn't become endemic, atleast in the first 5 years 
of the analysis. This may also be due to the large value of $M$, meaning the the exposed and 
infected mosquito populations are very small relatively to the susceptible population.
\\\\
For $k=5$ or 10, the sueceptible population rapidly decreases at 
a certain point in time, and the infected has a slight increase. In 5 years,
the population seems to become stable, however it would be possible to confirm 
if that's the case by increasing the maximum time of analysis.
\\\\
Now, we can see that we may double the infection rates, and the disease still 
won't become endemic. In the image below, it is shown how $\mathcal{R}_0$ 
increases as $k$ increases:
\begin{figure}[!ht]
        \centering
        \hbox{\hspace{2.5em} \includegraphics[scale=0.75] {THESIS-R0_vs_k.png}}
        \caption{$\mathcal{R}_0$ given in function of $k$}
\end{figure}
\newpage
So, using the parameters from Figures 7 and 8, we can see how the disease 
will only become endemic when $k>2.075$. Using this results, we can see that
deforestation does indeed cause a significant impact on the disease,
as increasing the biting rate in up to 5 to 10 times, starting with a single 
infected/exposed individual will lead the disease to infect a large portion 
of the human population, nearly $40\%$, before decreasing into a stable 
recovered population.
\\\\
In fact, we may analyze how the equilibriums of $S_H$ and $I_H$ are 
affected in function of $k$:
\begin{figure}[!ht]
        \centering
        \hbox{\hspace{5.0em} \includegraphics[scale=0.6] {THESIS-Equilibrium_IH_vs_k.png}}
        \caption{$I_H^*$ given in function of $k$}
\end{figure}
\begin{figure}[!ht]
        \centering
        \hbox{\hspace{5.0em} \includegraphics[scale=0.6] {THESIS-Equilibrium_SH_vs_k.png}}
        \caption{$S_H^*$ given in function of $k$}
\end{figure}
\\\\
By starting the plots for $k$ which makes $\mathcal{R}_0 = 1$ and for $t=1825$, 
as we want the final equilibrium, 
it is possible to see that the endemic equilibrium of infected 
humans approaches to a little over 60 as $k$ increases to 10. 
In fact, when $k=10$, $I_H^* \approx 61.64$. Analyzing the 
susceptible population, it decreases to values under 500, 
$S_H^* \approx 317$ when $k=10$ \footnote{These results can be found respectively at
https://github.com/RaphaLevy/Undergraduate\_Dissertation/blob/main/
\\modeling\_files/Plotting\_Equilibriums.ipynb.}.
\\\\ 
Iniciando o plot a partir de $k$ que deixa $\mathcal{R}_0 = 1$, é possível ver que
o equilíbrio endêmico da população humana se aproxima de 64 conforme $k$ se aproxima de 10.
Mais especificamente, quando $k=10$, $I_H^* \approx 63.49$ \footnote{A elaboração 
dos gráficos em função de $k$ podem ser encontrados em https://github.com/RaphaLevy/TCC/blob/main/Modelagem\_com\_Dinamica\_Pop/Plota\_Equilibrio\_e\_R0.ipynb. 
O cálculo dos equilíbrios pode ser encontrado em
https://github.com/RaphaLevy/TCC/blob/main/Modelagem\_com\_Dinamica\_Pop/R0\_com\_Dinamicas\_Demograficas.ipynb}.
Analisando o equilíbrio de suscetíveis conforme $k$ aumenta, é possível 
ver o equilíbrio decaindo rapidamente
de $N$ quando $k=0$ para aproximadamente 5000 indivíduos quando $k=1$. Analisando 
nos valores de $k$ tais que $\mathcal{R}_0 \geq 1$:
% \begin{figure}[!ht]
%         \centering
%         \hbox{\hspace{4.2em} \includegraphics[scale=0.7] {Plot_S_H_vs_k.png}}
%         \caption{$S_H^*$ em função de $k$}
% \end{figure} 
\newpage
Nesse caso, a população de suscetíveis tende a aproximadamente 95 conforme 
$k$ se aproxima de 10. Tendo calculado os equilíbrios de $S_H$ e $I_H$,
foi possível fazer uma análise de estabilidade global. Como estamos interessados
em analisar o equilíbrio endêmico, utilizei $k=10$:
\begin{figure}[!ht]
        \centering
        \hbox{\hspace{2.2em} \includegraphics[scale=0.7] {Equilibrio_SH_IH_k_10.png}}
        \caption{Equilíbrio global $S_H^* \times I_H^*$ para $k=10$}
\end{figure}
\\\\
Nesse caso, foram feitas 6 análises, a primeira utilizando os
valores iniciais de $S_H$ e $I_H$ como sendo 7716 e 1, e as demais aumentando $I_H$ 
em 1000 e diminuindo $S_H$ em 1000 indivíduos. Nesse caso, é possível
ver as populações de suscetíveis e infectados com um equilíbrio final de
aproximadamente 8 e 70 pessoas, respectivamente. Com isso,
poderíamos comparar o resultado obtido com o cálculo do equilíbrio endêmico de Adda e Bichara
\cite{adda2011global}, onde
\begin{gather*}
        S_H^* = \dfrac{1}{\mathcal{R}_0} \\
        I_H^* = \dfrac{\mu_H}{\mu_H+\gamma}(1-\dfrac{1}{\mathcal{R}_0})
\end{gather*}       
Através desse cálculo, a população de suscetíveis e infectados no equilíbrio 
foi de aproximadamente 1601 e 51, respectivamente. Notavelmente, esses valores 
estão destoantes dos 
obtidos através do cálculo numérico. Contudo, é necessário considerar a principal diferença entre as equações
propostas para $S$ e $I$ nesse Trabalho e no artigo de Adda e Bichara, que é o uso de
$I_M$ na taxa de infecção $\beta$, dada a dinâmica do modelo 
acoplado de SIR e SEI nesse caso, que não está sendo considerada no trabalho
de Adda e Bichara.

\chapter{Discussion}

Throughout the work, a SIR/SEI transmission model for malaria was 
developed and analyzed to complement the original methodology by 
Parham \& Michael. The model incorporated epidemiological factors of 
disease transmission and complemented them with human demographic dynamics 
and external environmental factors such as temperature and precipitation, 
including deforestation, considered in the multiplicative factor for the 
infection-causing bite proportions.
\\\\
Given the significant focus on modeling transmission and analyzing the 
environmental impacts, requiring various adaptations to make it more realistic 
and compatible with disease dynamics in the environment, the in-depth analysis 
of the effects of socioeconomic factors, as well as the identification of 
disease prevention and control strategies, ended up beyond the scope of the 
undergraduate thesis.
\\\\
Regarding what was accomplished, a key takeaway from the work is that 
this model is highly sensitive to the parameters used. Small modifications 
are sufficient for the model to reach equilibrium or for populations to tend 
toward $\pm \infty$.
\\\\
Indeed, what was particularly noted when modifying parameters 
$A, \ B,$ and $C$ was that using the values indicated in the 
works of Parham \& Michael \cite{Parham2010} and Eikenberry \& 
Gummel \cite{OKUNEYE201772}, in the case 
$A = -0.03, \ B = 1.31, \ C = -4.4$, no epidemic occurred. 
Starting the modeling with the other parameters used in 
Figure 7, with $k=1$, $\mathcal{R}_0 = 0.0207$, a value 
much lower than previously verified. Even with $k=10$, $\mathcal{R}_0 = 0.207$, 
leading to mosquito population extinction in all cases and making the 
existence of endemic equilibrium unviable.
\\\\
Another point that could be perceived was that in many cases, 
malaria takes a long time to reach endemic equilibrium. Therefore, 
although it was not possible to study the application of disease 
control strategies, it can be concluded that long-term measures 
may not be as effective, as the environmental conditions present 
at the beginning of this analysis would no longer be the same when 
the measure is applied.
\\\\
As future work, it would be possible to compare the general 
methodology used by the referenced authors with the methodology 
used in this work and identify what could be modified so that, 
using the original parameters, the model still reaches endemic equilibrium.
\\\\
Furthermore, as verified in the equilibrium calculations, 
something else that could be studied as a continuation of 
the work is the analysis of its evolution over time since 
they are given by oscillatory functions, such as the biting 
rate, birth rate, mortality rate, survival probability, and 
the infection rate of exposed individuals, which vary 
depending on temperature and precipitation factors, as presented previously. 
\\\\
Another possible option would be to analyze the use of $k$ in relation to other 
parameters instead of only the infection rate,
such as the mosquito's birth and mortality rates, daily survival rates, among others.


\chapter{Conclusion}

Throughout the development of the undergraduate thesis, 
different modifications to the dynamics of malaria transmission in the Amazon were explored 
in order to align the modeling more closely with the natural history of the disease in this 
environment. The ultimate goal was to understand how ecological impacts in the 
region affect interactions between the vector and the host.
\\\\
With the obtained results, it was possible to 
perceive the effect that increased contact between humans and mosquitoes due 
to deforestation can have on malaria dynamics, based on the proportion of bites 
causing infection. Furthermore, it was observed how, depending on the original 
parameters provided, a much higher proximity between vector and host would be 
required for the disease to become endemic in the Amazon region. As verified, 
this contact could be double the normal amount, and yet it is still insufficient 
for the disease to become an epidemic.
\\\\
To further align the methods used with observed behaviors in reality, 
the application of a stochastic transmission model, incorporating constantly 
changing environmental variables, could be ideal. However, for the proposed 
purpose, the deterministic model used was sufficient to highlight the disease's 
sensitivity to climate and environmental changes, allowing a clear and focused 
analysis of interactions between vector and host. This provides a solid foundation 
for investigating the implications of environmental changes on malaria 
transmission and for future research and improvements in the model.


%\input{apendices.tex}

% -----------------------------------
% ELEMENTOS PÓS-TEXTUAIS
% -----------------------------------
\postextual
% ----------------------------------

%\bibliography{biblio}
\printbibliography

%\glossary

% ----------------------------------------------------------
% Apêndices
% ----------------------------------------------------------

% ---
% Inicia os apêndices
% ---
\begin{apendicesenv}

% Imprime uma página indicando o início dos apêndices
\partapendices

\chapter{Results}
In this section, there will be included some resulting plots  
obtained of the development of the dissertation.
\\\\
% \begin{figure}[h]
% \end{figure}	

\begin{figure}[!ht]
	\centering
	\hbox{\hspace{2.0em} \includegraphics[scale=0.55] {THESIS-SIR_T_e_R_adaptados_CORRECAO.png}}
	\caption*{Appendix 1: SIR model with corrected $T$ and $R$}
\end{figure} 
\begin{figure}[!ht]
	\centering
	\hbox{\hspace{2.5em} \includegraphics[scale=0.55] {THESIS-SEI_T_e_R_adaptados_CORRECAO.png}}
	\caption*{Appendix 2: SEI model with corrected $T$ and $R$}
\end{figure} 
\newpage
\begin{figure}[!ht]
	\centering
	\hbox{\hspace{3.5em} \includegraphics[scale=0.55] {THESIS-SIR_RL_adaptado_CORRECAO.png}}
	\caption*{Appendix 3: SIR model with corrected $R_L$}
\end{figure} 
\begin{figure}[!ht]
	\centering
	\hbox{\hspace{3.5em} \includegraphics[scale=0.55] {THESIS-SEI_RL_adaptado_CORRECAO.png}}
	\caption*{Appendix 4: SEI model with corrected $R_L$}
\end{figure}
\newpage
\begin{figure}[!ht]
	\centering
	\hbox{\hspace{2.0em} \includegraphics[scale=0.55] {THESIS-SIR_Aumenta_ABC_CORRECAO.png}}
	\caption*{Appendix 5: SIR model with modified $A, \ B$ and $C$}
\end{figure} 
\begin{figure}[!ht]
	\centering
	\hbox{\hspace{2.0em} \includegraphics[scale=0.55] {THESIS-SEI_Aumenta_ABC_CORRECAO.png}}
	\caption*{Appendix 6: SEI model with modified $A, \ B$ and $C$}
\end{figure}
\newpage
\begin{figure}[!ht]
	\centering
	\hbox{\hspace{0.8em} \includegraphics[scale=0.55] {THESIS-SIR_Correcao_b3_T_linha_19_9.png}}
	\caption*{Appendix 7: SIR model with the inclusion of $b_3$ and $T'=19.9^\circ C$}
\end{figure} 
\begin{figure}[!ht]
	\centering
	\hbox{\hspace{0.8em} \includegraphics[scale=0.55] {THESIS-SEI_Correcao_b3_T_linha_19_9.png}}
	\caption*{Appendix 8: SEI model with the inclusion of $b_3$ and $T'=19.9^\circ C$}
\end{figure}
\newpage
\begin{figure}[!ht]
	\centering
	\hbox{\hspace{-1.0em} \includegraphics[scale=0.6] {THESIS-SIR_Correcao_b3_T_linha_24_4.png}}
	\caption*{Appendix 9: SIR model with the inclusion of $b_3$ and $T'=24.4^\circ C$}
\end{figure} 
\begin{figure}[!ht]
	\centering
	\hbox{\hspace{-1.0em} \includegraphics[scale=0.6] {THESIS-SEI_Correcao_b3_T_linha_24_4.png}}
	\caption*{Appendix 10: SEI model with the inclusion of $b_3$ and $T'=24.4^\circ C$}
\end{figure}
\newpage
\begin{figure}[!ht]
	\centering
	\hbox{\hspace{-1.2em} \includegraphics[scale=0.6] {THESIS-SIR_Correcao_l_b3_T_linha_19_9.png}}
	\caption*{Appendix 11: SIR model modifying $l$ and $T'=19.9^\circ C$}
\end{figure} 
\begin{figure}[!ht]
	\centering
	\hbox{\hspace{-1.0em} \includegraphics[scale=0.6] {THESIS-SEI_Correcao_l_b3_T_linha_19_9.png}}
	\caption*{Appendix 12: SEI model modifying $l$ and $T'=19.9^\circ C$}
\end{figure}
\newpage
\begin{figure}[!ht]
	\centering
	\hbox{\hspace{-1.2em} \includegraphics[scale=0.6] {THESIS-SIR_Correcao_l_b3_T_linha_24_4.png}}
	\caption*{Appendix 13: SIR model modifying $l$ and $T'=24.4^\circ C$}
\end{figure} 
\begin{figure}[!ht]
	\centering
	\hbox{\hspace{-1.0em} \includegraphics[scale=0.6] {THESIS-SEI_Correcao_l_b3_T_linha_24_4.png}}
	\caption*{Appendix 14: SEI model modifying $l$ and $T'=24.4^\circ C$}
\end{figure}
\newpage
\begin{figure}[!ht]
	\centering
	\hbox{\hspace{3.5em} \includegraphics[scale=0.8] {THESIS-Plot_a(T).png}}
	\caption*{Appendix 15: Biting rate $(a(T))$}
\end{figure} 
\begin{figure}[!ht]
	\centering
	\hbox{\hspace{3.5em} \includegraphics[scale=0.8] {THESIS-Plot_b3(T).png}}
	\caption*{Appendix 16: Exposed to infected rate $(b_3(T)$)}
\end{figure}
\newpage
\begin{figure}[!ht]
	\centering
	\hbox{\hspace{3.8em} \includegraphics[scale=0.8] {THESIS-Plot_tau_L(T).png}}
	\caption*{Appendix 17: Duration of larval development $(\tau_L(T))$}
\end{figure} 
\begin{figure}[!ht]
	\centering
	\hbox{\hspace{2.0em} \includegraphics[scale=0.8] {THESIS-Plot_p(T).png}}
	\caption*{Appendix 18: Daily mosquito survival rate $(p(T)$)}
\end{figure}
\newpage
\begin{figure}[!ht]
	\centering
	\hbox{\hspace{4.0em} \includegraphics[scale=0.8] {THESIS-Plot_p_LR(R).png}}
	\caption*{Appendix 19: Probability of larval survival $(p_{LR}(R))$}
\end{figure} 
\begin{figure}[!ht]
	\centering
	\hbox{\hspace{3.5em} \includegraphics[scale=0.8] {THESIS-Plot_p_LT(T).png}}
	\caption*{Appendix 20: Probability of larval survival $(p_{LT}(T)$)}
\end{figure}
\newpage
\begin{figure}[!ht]
	\centering
	\hbox{\hspace{4.0em} \includegraphics[scale=0.8] {THESIS-Plot_p_ER(R).png}}
	\caption*{Appendix 21: Probability of larval survival $(p_{ER}(R))$}
\end{figure} 
\begin{figure}[!ht]
	\centering
	\hbox{\hspace{4.0em} \includegraphics[scale=0.8] {THESIS-Plot_p_PR(R).png}}
	\caption*{Appendix 22: Probability of larval survival $(p_{PR}(R))$}
\end{figure}
\newpage
\begin{figure}[!ht]
	\centering
	\hbox{\hspace{4.0em} \includegraphics[scale=0.8] {THESIS-Plot_tau_M(T).png}}
	\caption*{Appendix 23: Duration of sporogonic cycle $(\tau_M(T))$}
\end{figure} 
\begin{figure}[!ht]
	\centering
	\hbox{\hspace{2.0em} \includegraphics[scale=0.8] {THESIS-Plot_mu(T).png}}
	\caption*{Appendix 24: Mosquito mortality rate $(\mu(T))$}
\end{figure}
\newpage
\begin{figure}[!ht]
	\centering
	\hbox{\hspace{4.0em} \includegraphics[scale=1.2] {THESIS-Plot_p_LRT(R_T).png}}
	\caption*{Appendix 25: Probability of larval survival $(p_{LRT}(R,T))$}
\end{figure} 
\begin{figure}[!ht]
	\centering
	\hbox{\hspace{4.0em} \includegraphics[scale=1.2] {THESIS-Plot_b(R_T).png}}
	\caption*{Appendix 26: Mosquito mortality rate $(b(R,T))$}
\end{figure}
\newpage
\begin{figure}[!ht]
	\centering
	\hbox{\hspace{3.0em} \includegraphics[scale=0.6] {THESIS-SIR_k_1_5.png}}
	\caption*{Appendix 27: SIR model with $k=1.5$}
\end{figure} 
\begin{figure}[!ht]
	\centering
	\hbox{\hspace{3.0em} \includegraphics[scale=0.6] {THESIS-SEI_k_1_5.png}}
	\caption*{Appendix 28: SEI model with $k=1.5$}
\end{figure}
\newpage
\begin{figure}[!ht]
	\centering
	\hbox{\hspace{3.0em} \includegraphics[scale=0.6] {THESIS-SIR_k_2.png}}
	\caption*{Appendix 29: SIR model with $k=2$}
\end{figure} 
\begin{figure}[!ht]
	\centering
	\hbox{\hspace{3.0em} \includegraphics[scale=0.6] {THESIS-SEI_k_2.png}}
	\caption*{Appendix 30: SEI model with $k=2$}
\end{figure}
\newpage
\begin{figure}[!ht]
	\centering
	\hbox{\hspace{3.0em} \includegraphics[scale=0.6] {THESIS-SIR_k_2_5.png}}
	\caption*{Appendix 31: SIR model with $k=2.5$}
\end{figure} 
\begin{figure}[!ht]
	\centering
	\hbox{\hspace{3.0em} \includegraphics[scale=0.6] {THESIS-SEI_k_2_5.png}}
	\caption*{Appendix 32: SEI model with $k=2.5$}
\end{figure}
\newpage
\begin{figure}[!ht]
	\centering
	\hbox{\hspace{3.0em} \includegraphics[scale=0.6] {THESIS-SIR_k_5.png}}
	\caption*{Appendix 33: SIR model with $k=5$}
\end{figure} 
\begin{figure}[!ht]
	\centering
	\hbox{\hspace{3.0em} \includegraphics[scale=0.6] {THESIS-SEI_k_5.png}}
	\caption*{Appendix 34: SEI model with $k=5$}
\end{figure}
\newpage
\begin{figure}[!ht]
	\centering
	\hbox{\hspace{3.0em} \includegraphics[scale=0.6] {THESIS-SIR_k_10.png}}
	\caption*{Appendix 35: SIR model with $k=10$}
\end{figure} 
\begin{figure}[!ht]
	\centering
	\hbox{\hspace{3.0em} \includegraphics[scale=0.6] {THESIS-SEI_k_10.png}}
	\caption*{Appendix 36: SEI model with $k=10$}
\end{figure}
\newpage
\begin{figure}[!ht]
	\centering
	\hbox{\hspace{7.0em} \includegraphics[scale=0.65] {THESIS-SH_for_k_values_Correction.png}}
	\caption*{Appendix 37: Susceptible human population for varying values of $k$}
\end{figure} 
\begin{figure}[!ht]
	\centering
	\hbox{\hspace{7.0em} \includegraphics[scale=0.65] {THESIS-IH_for_k_values_Correction.png}}
	\caption*{Appendix 38: Infected human population for varying values of $k$}
\end{figure}
\begin{figure}[!ht]
	\centering
	\hbox{\hspace{7.0em} \includegraphics[scale=0.65] {THESIS-RH_for_k_values.png}}
	\caption*{Appendix 39: Recovered human population for varying values of $k$}
\end{figure}
\newpage
\begin{figure}[!ht]
	\centering
	\hbox{\hspace{7.0em} \includegraphics[scale=0.65] {THESIS-SM_for_k_values.png}}
	\caption*{Appendix 40: Susceptible mosquito population for varying values of $k$}
\end{figure} 
\begin{figure}[!ht]
	\centering
	\hbox{\hspace{7.0em} \includegraphics[scale=0.65] {THESIS-EM_for_k_values.png}}
	\caption*{Appendix 41: Exposed mosquito population for varying values of $k$}
\end{figure}
\begin{figure}[!ht]
	\centering
	\hbox{\hspace{7.0em} \includegraphics[scale=0.65] {THESIS-IM_for_k_values.png}}
	\caption*{Appendix 42: Infected mosquito population for varying values of $k$}
\end{figure}
\newpage
\begin{figure}[!ht]
	\centering
	\hbox{\hspace{7.0em} \includegraphics[scale=0.65] {THESIS-SH_k=2_075.png}}
	\caption*{Appendix 43: Susceptible human population for varying values of $I_{H0}, \ k \approx 2.075$}
\end{figure}
\begin{figure}[!ht]
	\centering
	\hbox{\hspace{7.0em} \includegraphics[scale=0.65] {THESIS-IH_k=2_075.png}}
	\caption*{Appendix 44: Infected human population for varying values of $I_{H0}, \ k \approx 2.075$}
\end{figure}
\newpage
\begin{figure}[!ht]
	\centering
	\hbox{\hspace{7.0em} \includegraphics[scale=0.65] {THESIS-SH_k=10.png}}
	\caption*{Appendix 45: Susceptible human population for varying values of $I_{H0}, \ k=10$}
\end{figure}
\begin{figure}[!ht]
	\centering
	\hbox{\hspace{7.0em} \includegraphics[scale=0.65] {THESIS-IH_k=10.png}}
	\caption*{Appendix 46: Infected human population for varying values of $I_{H0}, \ k=10$}
\end{figure}
\newpage





\end{apendicesenv}
% ---

% ----------------------------------------------------------
% Anexos
% ----------------------------------------------------------

% \begin{anexosenv}

% \partanexos

% \end{anexosenv}
%"
%---------------------------------------------------------------------
% ÍNDICE REMISSIVO
%---------------------------------------------------------------------
\phantompart
\printindex

\end{document}