\chapter{Introduction}

The Amazon is one of the largest and most biodiverse tropical forests 
in the world, harboring numerous species of plants, animals, and 
microorganisms, including vectors and pathogens responsible for the 
transmission of various diseases. Among them, one of the most common 
is malaria, caused by protozoa of the genus \textit{Plasmodium}, 
transmitted by the bite of the infected female mosquito of the genus 
\textit{Anopheles}. It is present in 22 American countries, but the 
areas with the highest risk of infection are located in the Amazon 
region, encompassing nine countries, which accounted for $68\%$ of 
infection cases in 
2011 \cite{pimenta_orfano_bahia_duarte_rios-velasquez_melo_pessoa_oliveira_campos_villegas_etal_2015}. 
Although malaria is prevalent in the Americas, it is 
not limited to this continent and is found in countries in Africa and Asia, 
resulting in more than two million cases of infection and 445,000 deaths 
worldwide in 2016 \cite{doi:10.1146/annurev-micro-090817-062712}.
\\\\
Notably, vector-borne disease transmission is closely related to 
environmental changes that interfere with the ecosystem of both 
transmitting organisms and affected organisms. In the case of the 
Amazon, agricultural and livestock settlements are among the factors
that most favor disease transmission, both due to the deforestation 
they cause for establishment and the clustering of people in environments 
close to the vector's habitat \cite{silva-nunes_malaria_amazon_2008}, 
especially by clustering non-immune migrants near these natural and 
artificial breeding sites \cite{DASILVANUNES2012281}.
\\\\
Additionally, other factors such as rainfall, wildfires, and mining 
also significantly influence disease transmission in the region. These 
events result in habitat loss, ecosystem fragmentation, and climate 
changes, affecting the distribution and abundance of vectors and hosts, 
as well as their interaction with pathogens. Furthermore, population growth 
and urbanization also play a crucial role in disease spread, increasing 
human exposure to vectors and infection risks.
\\\\
In this context, this work aims to investigate vector-borne disease 
transmission in the Amazon and analyze how environmental impacts 
influence the dynamics of malaria transmission, the ecological and 
socioeconomic factors affecting this spread, and possible prevention 
and control strategies. The main reference for this research is the 
Trajetórias Project, developed by the Center for Biodiversity and 
Ecosystem Services (SinBiose/CNPq), which is a dataset including 
environmental, epidemiological, economic, and socioeconomic indicators 
for all municipalities in the Legal Amazon, analyzing the spatial and 
temporal relationship between economic trajectories linked to the dynamics 
of agrarian systems, whether they are family-based rural or large-scale 
agricultural and livestock production, the availability of natural resources, 
and the risk of diseases \cite{Rorato2023}.