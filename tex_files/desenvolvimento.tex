\chapter{Metodology}

For the elaboration of the work, population data from the 
Trajetórias Project dataset and climatic data from the Climate 
Data will be used. Methods of disease transmission based on ordinary 
differential equations, such as the SIR model, will be addressed. 
Starting with a simple modeling approach, environmental phenomena 
such as deforestation and burning will be included to assess how 
modifications in the ecosystem will interfere with the previously 
developed model. Computational calculations were performed using 
the SageMath 9.2 environment, utilizing Scipy's numerical 
integration functions to solve the method.
\\\\
First I will be describing SIR \cite{githubMODBIO, Prasad2022}, 
which can be considered the foundation 
of the models that will be used throughout the project. 
Developed by W. O. Kermack and A. G. McKendrick in 1927, 
SIR is one of the most widely used models for epidemic 
modeling, considering three compartments:

\begin{align*}
    & S: \text{number of susceptible individuals} \\
    & I: \text{number of infected individuals} \\
    & R: \text{number of recovered individuals}
\end{align*}
\\
In this model, healthy individuals in the $S$ class are 
susceptible to contact with individuals in the $I$ class and 
are transferred to this compartment if they contract the disease. 
Infected individuals can spread the disease through 
direct contact with susceptible individuals, but they can 
also become immune over time and are transferred to the $R$ 
compartment. In general, $R$ includes the total of recovered (immune) 
individuals and those who died from the disease, but we can assume 
that the number of deaths is very low compared to the total 
population size and can be ignored. We also assume that individuals 
in this category will not revert to being susceptible or infectious.
\\\\
Considering an epidemic over a short period and 
that the disease is not fatal, we can ignore vital 
dynamics of birth and death. With this, we can describe 
the SIR model through the following system of ordinary 
differential equations (ODEs):

\begin{gather*}
\begin{cases}
\dfrac{dS}{dt} = -\dfrac{\beta SI}{N} \\
\\
\dfrac{dI}{dt} = \dfrac{\beta SI}{N} - \gamma I \\
\\
\dfrac{dR}{dt} = \gamma I
\end{cases}
\end{gather*}
\\
In the model, $N(t) = S(t) + I(t) + R(t)$, i.e., the total population 
at time $t$, while $\beta$ is the infection rate, and $\gamma$ is the
recovery rate. Given that $S + I + R$ is always constant if we 
ignore birth and death, we have 
$\dfrac{dS}{dt} + \dfrac{dI}{dt} + \dfrac{dR}{dt} = 0$.
\\\\
For the disease to spread, it is easy to see that 
$\dfrac{dI}{dt} = \dfrac{\beta SI}{N} - \gamma I > 0$. 
Thus, $\dfrac{\beta SI}{N} > \gamma I \Rightarrow \dfrac{\beta S}{N} > \gamma$. 
Assuming we are at the beginning of the infection, given 
that we want to observe its spread, $I$ will be very small, 
and $S \approx N$. We then conclude that 
$\dfrac{\beta N}{N} > \gamma \Rightarrow \dfrac{\beta}{\gamma} > 1$. 
This dimensionless value can be derived by nondimensionalizing 
the model: 
let $y^* = \dfrac{S}{N}$, $x^* = \dfrac{I}{N}$, $z^* = \dfrac{R}{N}$, 
and $t^* = \dfrac{t}{1/\gamma} = \gamma t$, so that $y^* + x^* + z^* = 1$. 
Substituting the system of ODEs above using these values:


\begin{gather*}
\begin{cases}
\dfrac{dS}{dt} = \dfrac{d(y^*N)}{d(t^*/\gamma)} = -\dfrac{\beta SI}{N} = -\dfrac{\beta(y^*N)(x^*N)}{N} = -\beta y^*Nx^* \\
\\
\dfrac{dI}{dt} = \dfrac{d(x^*N)}{d(t^*/\gamma)} = \dfrac{\beta SI}{N} - \gamma I = \dfrac{\beta(y^*N)(x^*N)}{N} -\gamma(x^*N) = \beta y^*Nx^* - \gamma x^*N \\
\\
\dfrac{dR}{dt} = \dfrac{d(z^*N)}{d(t^*/\gamma)} = \gamma I = \gamma(x^*N)
\end{cases}
\end{gather*}

Now, canceling the factors $N$ and $\gamma$ on both sides of the equations:

\begin{gather*}
\begin{cases}
\dfrac{d(y^*)}{d(t^*)} = -\dfrac{\beta y^*x^*}{\gamma} \\
\\
\dfrac{d(x^*)}{d(t^*)} = \dfrac{\beta y^*x^*}{\gamma} - x^* \\
\\
\dfrac{d(z^*)}{d(t^*)} = x^*
\end{cases}
\end{gather*}
\\
Thus, we have a system given only by $y^*$ and $x^*$ and the 
parameter $\dfrac{\beta}{\gamma}$, which we can call $R_0$.
\\\\
As this work will be primarily focused on malaria modeling, 
I will now present one of the first models developed specifically 
for this disease, by Sir Ronald Ross in 1911 \cite{Bacaër2011}, 
which uses two distinct ordinary differential equations (ODEs) 
different from those presented above:

\begin{gather*}
\begin{cases}
\dfrac{dI}{dt} = bp'i\dfrac{N-I}{N} -aI\\
\\
\dfrac{di}{dt} = bp(n-i)\dfrac{I}{N} - mI
\end{cases}
\end{gather*}
\\
In this case, $N$ is the total human population, $I(t)$ is the number 
of infected humans at time $t$, $n$ is the total mosquito population, 
$i(t)$ is the number of infected mosquitoes at time $t$, $b$ is the 
biting rate, $p$ is the probability of transmission from human to 
mosquito per bite, $p'$ is the probability of transmission from 
mosquito to human per bite, $a$ is the recovery rate of human infection, 
and $m$ is the mosquito mortality rate. $bp'ii\dfrac{N-I}{N}dt -aIdt$ 
represent, respectively, the number of new infected humans and 
the number of recovered humans in the interval $dt$, while 
$bp(n-i)\dfrac{I}{N}dt - mIdt$ represent, respectively, the 
number of new infected mosquitoes and the number of mosquitoes that 
die in that time interval, assuming that infection does not affect 
the mosquito mortality rate.
\\\\
For this model, Ross discussed two equilibrium points, where 
$\dfrac{dI}{dt} = \dfrac{di}{dt} = 0$. They occur when $I=i=0$, 
which is the case where there is no malaria, and, for $I, i > 0$, 
$I = N\dfrac{1-amN/(b^2pp'n)}{1+aN/(bp'n)}$ and 
$i = n\dfrac{1-amN/(b^2pp'n)}{1+m/(bp)}$. Furthermore, for the 
disease to establish itself, $n$ must be greater than a 
threshold value $n^* = \dfrac{amN}{b^2pp'}$. In this case, 
the disease becomes endemic. If $n<n^*$, the equilibrium will 
be at $I=i=0$, and the disease will disappear.
\\\\
Dividing the equations of the equilibrium points by $I \times i$, we have:

\begin{gather*}
\begin{cases}
\dfrac{bp}{N} = \dfrac{bpn}{Ni} -\dfrac{m}{I} \\
\\
\dfrac{bp'}{N} = \dfrac{bp'}{I} -\dfrac{a}{i} 
\end{cases}
\end{gather*}
\\
Which transforms the problem into a linear 
system with two unknowns, $I$ and $i$.
\\\\
Now, I will present the model that will be used 
for the development of the work, based on the one 
developed by Paul E. Parham and Edwin Michael in 
2010, which takes into account factors such as 
rainfall and temperature ($R$ and $T$, respectively) \cite{Parham2010}.
\\\\
Defining the equations that will be used:
\begin{gather*}
\begin{cases}
\dfrac{dS_H}{dt} = -ab_2\bigg(\dfrac{I_M}{N}\bigg)S_H\\
\\
\dfrac{dI_H}{dt} = ab_2\bigg(\dfrac{I_M}{N}\bigg)S_H-\gamma I_H\\
\\
\dfrac{dR_H}{dt} = \gamma I_H\\
\\
\dfrac{dS_M}{dt} = b - ab_1\bigg(\dfrac{I_H}{N}\bigg)S_M - \mu S_M\\
\\
\dfrac{dE_M}{dt} = ab_1\bigg(\dfrac{I_H}{N}\bigg)S_M - \mu E_M - ab_1\bigg(\dfrac{I_H}{N}\bigg)S_Ml(\tau_M)\\
\\
\dfrac{dI_M}{dt} = ab_1\bigg(\dfrac{I_H}{N}\bigg)S_Ml(\tau_M) -\mu I_M
\end{cases}
\end{gather*}
\\\\
It is necessary to mention that the original model 
used $I_M(t-\tau)$ in $\dfrac{dI_H}{dt}$ and $I_H(t-\tau)$ in 
$\dfrac{dE_M}{dt}$ (in the transition from $E$ to $I$) and 
$\dfrac{dI_M}{dt}$, respectively. However, as this would 
make the model based on delay differential equations, it was 
recommended by the advisor to disregard this difference and use 
only the current time $t$.
\\\\
Having the model equations for the human and mosquito 
populations, I will first define the parameters used in the modeling 
and other necessary functions, and then the variables used:
\\
\begin{adjustwidth}{-0.5cm}{}
    \begin{center}
    \renewcommand{\arraystretch}{1.5}
    \raggedleft\begin{tabular}{|c | l | c|} 
     \hline
     \raisebox{-1ex}{\textbf{Parameter}} & \raisebox{-1ex}{\textbf{Definition}} & \raisebox{-1ex}{\textbf{Formula}}\\ 
     \hline
     $T(t)$ & \pbox{8cm}{\rule{0pt}{4.5ex}Temperature\rule[-2.5ex]{0pt}{0pt}} & $T_1 (1 + T_2 \cos(\omega_1t - \phi_1))$\\ 
     \hline
     $R(t)$ & \pbox{8cm}{\rule{0pt}{4.5ex}Precipitation\rule[-2.5ex]{0pt}{0pt}} & $R_1 (1 + R_2 \cos(\omega_2t - \phi_2))$ \\
     \hline
     $b(R, T)$ & \pbox{8cm}{\rule{0pt}{4.5ex}Mosquito birth rate (/ day)\rule[-2.5ex]{0pt}{0pt}} & $\dfrac{B_E  p_E(R)  p_L(R,T)  p_P(R)}{(\tau_E + \tau_L(T) + \tau_P)}$\\ 
     \hline
     $a(T)$ & \pbox{8cm}{\rule{0pt}{4.5ex}Biting rate (/day)\rule[-2.5ex]{0pt}{0pt}} & $\dfrac{(T - T_1)}{D_1}$ \\
     \hline
     $\mu(T)$ & \pbox{8cm}{\rule{0pt}{3ex}Mosquito mortality rate per capita (/ day)\rule[-1.5ex]{0pt}{0pt}} & $-\log(p(T))$ \\
     \hline
     $\tau_M(T)$ & \pbox{8cm}{\rule{0pt}{4.5ex}Duration of the sporozoite cycle (days)\rule[-2.5ex]{0pt}{0pt}} & $\dfrac{DD}{(T - T_{min})}$ \\
     \hline
     $\tau_L(T)$ & \pbox{8cm}{\rule{0pt}{4.5ex}Duration of larval development phase (days)\rule[-2.5ex]{0pt}{0pt}} & $\dfrac{1}{c_1T + c_2}$ \\
     \hline
     $p(T)$ & \pbox{8cm}{\rule{0pt}{3ex}Daily mosquito survival rate\rule[-1.5ex]{0pt}{0pt}} & $e^{(-1 / (AT^2 + BT + C))}$ \\
     \hline
     $p_L(R)$ & \pbox{8cm}{\rule{0pt}{3ex}Probability of larval survival dependent on rainfall\rule[-1.5ex]{0pt}{0pt}} & $(\dfrac{4p_{ML}}{R_L^2})R(R_L - R)$ \\
     \hline
     $p_L(T)$ & \pbox{8cm}{\rule{0pt}{3ex}Probability of larval survival dependent on temperature\rule[-1.5ex]{0pt}{0pt}} & $e^{-(c_1T + c_2)}$ \\
     \hline
     $p_L(R, T)$ & \pbox{8cm}{\rule{0pt}{3ex}Probability of larval survival dependent on temperature and rainfall\rule[-1.5ex]{0pt}{0pt}} & $p_L(R)p_L(T)$ \\
     \hline
     $l(\tau_M)(T)$ & \pbox{8cm}{\rule{0pt}{3ex}Probability of mosquito survival during the sporozoite cycle (/ day)\rule[-1.5ex]{0pt}{0pt}} & $p(T)^{\tau_M(T)}$ \\
     \hline
     $M(t)$ & \pbox{8cm}{\rule{0pt}{3ex}Total number of mosquitoes\rule[-1.5ex]{0pt}{0pt}} & $S_M(t) + E_M(t) + I_M(t)$ \\
     \hline
     $N(t)$ & \pbox{8cm}{\rule{0pt}{3ex}Total number of humans\rule[-1.5ex]{0pt}{0pt}} & $S_H(t) + I_H(t) + R_H(t)$ \\  
     \hline
    \end{tabular}
    \captionof{table}{Parameters used in the modeling}
    \end{center}
    \end{adjustwidth}

    \begin{adjustwidth}{-0.5cm}{}
        \begin{center}
        \renewcommand{\arraystretch}{1.5}
        \raggedleft\begin{tabular}{|c | c|} 
         \hline
         \textbf{Parameter} & \textbf{Definition}\\ 
         \hline
         $b_1$ & \makecell[l]{\rule{0pt}{3ex}Proportion of bites from susceptible mosquitoes \\ on infected humans that result in infection\rule[-1.5ex]{0pt}{0pt}} \\
         \hline
         $b_2$ & \makecell[l]{\rule{0pt}{3ex}Proportion of bites from infected mosquitoes \\ on susceptible humans that result in infection\rule[-1.5ex]{0pt}{0pt}} \\
         \hline
         $\gamma$ & \makecell[l]{\rule{0pt}{3ex}1/Average duration of infectiousness in humans (days$^{-1}$)\rule[-1.5ex]{0pt}{0pt}} \\
         \hline
         $T_1$ & \makecell[l]{\rule{0pt}{3ex}Mean temperature in the absence of seasonality ($^\circ C$)\rule[-1.5ex]{0pt}{0pt}} \\
         \hline
         $T_2$ & \makecell[l]{\rule{0pt}{3ex}Amplitude of seasonal variability in temperature\rule[-1.5ex]{0pt}{0pt}} \\
         \hline
         $R_1$ & \makecell[l]{\rule{0pt}{3ex}Average monthly precipitation in the absence of \\ seasonality (mm)\rule[-1.5ex]{0pt}{0pt}} \\
         \hline
         $R_2$ & \makecell[l]{\rule{0pt}{3ex}Amplitude of seasonal variability in precipitation\rule[-1.5ex]{0pt}{0pt}} \\
         \hline
         $\omega_1$ & \makecell[l]{\rule{0pt}{3ex}Angular frequency of seasonal oscillations in temperature (months$^{-1}$)\rule[-1.5ex]{0pt}{0pt}} \\
         \hline
         $\omega_2$ & \makecell[l]{\rule{0pt}{3ex}Angular frequency of seasonal oscillations in precipitation (months$^{-1}$)\rule[-1.5ex]{0pt}{0pt}} \\
         \hline
         $\phi_1$ & \makecell[l]{\rule{0pt}{3ex}Phase lag of temperature variability (phase shift)\rule[-1.5ex]{0pt}{0pt}} \\
         \hline
         $\phi_2$ & \makecell[l]{\rule{0pt}{3ex}Phase lag of precipitation variability (phase shift)\rule[-1.5ex]{0pt}{0pt}} \\
         \hline
         $B_E$ & \makecell[l]{\rule{0pt}{3ex}Number of eggs laid per adult per oviposition\rule[-1.5ex]{0pt}{0pt}} \\
         \hline
         $p_{ME}$ & \makecell[l]{\rule{0pt}{3ex}Maximum probability of egg survival\rule[-1.5ex]{0pt}{0pt}} \\
         \hline
         $p_{ML}$ & \makecell[l]{\rule{0pt}{3ex}Maximum probability of larval survival\rule[-1.5ex]{0pt}{0pt}} \\
         \hline
         $p_{MP}$ & \makecell[l]{\rule{0pt}{3ex}Maximum probability of pupal survival\rule[-1.5ex]{0pt}{0pt}} \\
         \hline
         $\tau_E$ & \makecell[l]{\rule{0pt}{3ex}Duration of the egg development phase (days)\rule[-1.5ex]{0pt}{0pt}} \\
         \hline
         $b_3^*$ & \makecell[l]{\rule{0pt}{3ex}Infection rate in exposed mosquitoes $(1/\tau_M(T))$\rule[-1.5ex]{0pt}{0pt}} \\
         \hline
        \end{tabular}
        \captionof{table}{Parameters used in the modeling}
        \end{center}
        \end{adjustwidth}


        \begin{adjustwidth}{-0.5cm}{}
            \begin{center}
            \renewcommand{\arraystretch}{1.5}
            \raggedleft\begin{tabular}{|c | c|} 
             \hline
             \textbf{Parameter} & \textbf{Definition}\\ 
             \hline
              $\tau_P$ & \makecell[l]{\rule{0pt}{3ex}Duration of the pupal development phase (days)\rule[-1.5ex]{0pt}{0pt}} \\
             \hline
             $R_L$ & \makecell[l]{\rule{0pt}{3ex}Rainfall threshold until breeding sites are eliminated, \\ removing immature individuals (mm)\rule[-1.5ex]{0pt}{0pt}} \\
             \hline
             $T_{min}$ & \makecell[l]{\rule{0pt}{3ex}Minimum temperature, below which there is no development \\ of the parasite: 14.5 ($^\circ C$)\rule[-1.5ex]{0pt}{0pt}} \\
             \hline
             $DD$ & \makecell[l]{\rule{0pt}{3ex}Degree-days for parasite development. Number of degrees \\ by which the daily average temperature exceeds the minimum \\ development temperature.
             "Sum of heat" for maturation: 105 ($^\circ C \ \text{days}$) \rule[-1.5ex]{0pt}{0pt}} \\
             \hline
             $A$ & \makecell[l]{\rule{0pt}{3ex}Empirical sensitivity parameter ($^\circ C^2 \ \text{days})^{-1}$\rule[-1.5ex]{0pt}{0pt}} \\
             \hline
             $B$ & \makecell[l]{\rule{0pt}{3ex}Empirical sensitivity parameter ($^\circ C \ \text{days})^{-1}$\rule[-1.5ex]{0pt}{0pt}} \\
             \hline
             $C$ & \makecell[l]{\rule{0pt}{3ex}Empirical sensitivity parameter ($\text{days}^{-1}$)\rule[-1.5ex]{0pt}{0pt}} \\
             \hline
             $D_1$ & \makecell[l]{\rule{0pt}{3ex}Constant: 36.5 ($^\circ C \ \text{days}$)\rule[-1.5ex]{0pt}{0pt}} \\
             \hline
             $c_1$ & \makecell[l]{\rule{0pt}{3ex}Empirical sensitivity parameter ($^\circ C \ \text{days})^{-1}$\rule[-1.5ex]{0pt}{0pt}} \\
             \hline
             $c_2$ & \makecell[l]{\rule{0pt}{3ex}Empirical sensitivity parameter ($\text{days}^{-1}$)\rule[-1.5ex]{0pt}{0pt}} \\
             \hline
             $T'^*$ & \makecell[l]{\rule{0pt}{3ex}Empirical temperature parameter ($^\circ C$)\rule[-1.5ex]{0pt}{0pt}} \\
             \hline
            \end{tabular}
            \captionof{table}{Parameters used in the modeling}
            \end{center}
            \end{adjustwidth}

\vspace{1cm}
Parâmetros marcados com $^*$ foram adicionados durante o desenvolvimento da modelagem para correção de imprecisões derivadas das equações originais do artigo de referência. 
A definição de $DD$ foi retirada de \cite{McCord2016} e \cite{10665-41724}.
\\\\
Tendo as equações e parâmetros, a modelagem foi feita inicialmente 
utilizando dados da zona rural de Manaus, no período de 2004 a 2008, 
que foram selecionados devido à maior incidência de casos de malária causados 
por $P. \ vivax$, sendo a espécie responsável pelo maior número de
casos no Brasil \cite{OliveiraFerreira2010, 10.3389/fpubh.2021.647754}. Usando a função de incidência utilizada no projeto 
Trajetorias \cite{Rorato2023}, temos:
\begin{gather*}
    \text{Inc}(d, m, z, t_1, t_2) = \dfrac{\text{Casos}(d, m, z, t_1, t_2)}{\text{Pop}(m,z,(t_1+t_2)/2) \times 5 \ \text{anos}} \times 10^5,
\end{gather*}
onde $\text{casos}(d, m, z, t_1, t_2)$ é o número de casos da doença $d$ na zona $z$ do município $m$, e $t_1$ e $t_2$ são os anos iniciais e finais do intervalo, enquanto que $\text{pop}(m,z,(t_1+t_2)/2) \times 5 \ \text{anos}$ é a população na zona $z$ do município $m$ no meio do período multiplicado pelo total de anos de observação. Nesse caso, poderíamos indicar como:
\begin{gather*}
    \footnotesize{\text{Inc}(\text{Vivax}, \text{Manaus}, \text{Rural}, 2004, 2008) = \dfrac{\text{Casos}(\text{Vivax}, \text{Manaus}, \text{Rural}, 2004, 2008)}{\text{Pop}(\text{Manaus}, \text{Rural}, 2006) \times 5 \ \text{anos}} \times 10^5}  \\\\
    184030.8 = \dfrac{78745}{5\text{Pop}} \times 10^5 \Rightarrow Pop \approx 8558
\end{gather*}
Usando dados da população total de Manaus nesse período, cuja incidência foi de 3106.4 e o número de casos foi de 262264, a população total do município foi estimada como sendo de 1688540 habitantes. Com isso, a população rural pôde ser considerada como aproximadamente 0.5$\%$ da população do município.
\\\\
Tendo estimado o tamanho porcentual da população rural na cidade, 
foi possível calcular essa população em cada um dos anos da análise 
através de uma interpolação linear feita com dados de séries históricas 
do IBGE \cite{popIBGE}:
\\\\
\begin{adjustwidth}{0cm}{}
\begin{center}
\renewcommand{\arraystretch}{1.5}
\begin{tabular}{|c | c|} 
 \hline
 \textbf{Ano} & \textbf{População rural estimada}\\ 
 \hline
$2004$ & $7717$ \\
 \hline
 $2005$ & $7889$ \\
 \hline
 $2006$ & $8061$ \\
 \hline
 $2007$ & $8233$ \\
 \hline
 $2008$ & $8492$ \\
 \hline
 $2009$ & $8751$ \\
 \hline
\end{tabular}
\captionof{table}{População rural de Manaus de 2004 a 2009}
\end{center}
\end{adjustwidth}

\vspace{1cm}
Como haviam dados populacionais para os anos de 2000, 2007 e 2010, as interpolações foram feitas com diferentes pontos iniciais e finais, usando de 2000 a 2007 para 2004-2007, e de 2007 a 2010 para 2008-2009, de forma a utilizar corretamente a população de 2007.
\\\\
Descrevendo agora um pouco da teoria por trás dos fatores ambientais, 
segundo \cite{Norris2004}, 
a remoção da copa das árvores permitiu a reemergência da malária 
na América do Sul, já que em áreas desmatadas, 
sem as copas cobrindo o solo, poças d'água sob luz solar atraem mosquitos 
da espécie $Anopheles \ darlingi$, principal vetor relacionado à malária 
humana na Amazônia \cite{infoAnopheles}, sendo que costumam ser menos encontrados em 
florestas ainda intactas. Isso ocorre porque a luz e calor favorecem o 
desenvolvimento de larvas e pupas, além de uma maior disponibilidade de 
algas para alimentação das larvas \cite{article_alteracoesambientais}. O aumento da temperatura 
ambiente também favorece a capacidade vetorial dos mosquitos. O desmatamento 
também atrai e aproxima humanos para que possam tomar parte em atividades 
madeireiras, de agricultura e construção de rodovias, trazendo indíviduos 
infectados com o $Plasmodium$ para uma área em que tanto o vetor quanto o 
ambiente já foram modificados de forma a favorecer a sua transmissão. 
Ainda, a agricultura também favorece a sedimentação dos rios, sendo 
ambientes propícios para o estabelecimento de criadouros. Sendo assim, 
pode ser considerada uma mudança adequada para o modelo para levar 
em consideração o desmatamento, o aumento das probabilidades de 
sobrevivência dos ovos, larvas e pupas, além de aumentar a proporção 
de picadas que produzem infecção, devido ao aumento da densidade 
populacional humana em áreas próximas aos criadouros dos mosquitos.