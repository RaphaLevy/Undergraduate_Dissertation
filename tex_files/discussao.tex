\chapter{Discussion}

Throughout the work, a SIR/SEI transmission model for malaria was 
developed and analyzed to complement the original methodology by 
Parham \& Michael. The model incorporated epidemiological factors of 
disease transmission and complemented them with human demographic dynamics 
and external environmental factors such as temperature and precipitation, 
including deforestation, considered in the multiplicative factor for the 
infection-causing bite proportions.
\\\\
Given the significant focus on modeling transmission and analyzing the 
environmental impacts, requiring various adaptations to make it more realistic 
and compatible with disease dynamics in the environment, the in-depth analysis 
of the effects of socioeconomic factors, as well as the identification of 
disease prevention and control strategies, ended up beyond the scope of the 
undergraduate thesis.
\\\\
Regarding what was accomplished, a key takeaway from the work is that 
this model is highly sensitive to the parameters used. Small modifications 
are sufficient for the model to reach equilibrium or for populations to tend 
toward $\pm \infty$.
\\\\
Indeed, what was particularly noted when modifying parameters 
$A, \ B,$ and $C$ was that using the values indicated in the 
works of Parham \& Michael \cite{Parham2010} and Eikenberry \& 
Gummel \cite{OKUNEYE201772}, in the case 
$A = -0.03, \ B = 1.31, \ C = -4.4$, no epidemic occurred. 
Starting the modeling with the other parameters used in 
Figure 7, with $k=1$, $\mathcal{R}_0 = 0.0207$, a value 
much lower than previously verified. Even with $k=10$, $\mathcal{R}_0 = 0.207$, 
leading to mosquito population extinction in all cases and making the 
existence of endemic equilibrium unviable.
\\\\
Another point that could be perceived was that in many cases, 
malaria takes a long time to reach endemic equilibrium. Therefore, 
although it was not possible to study the application of disease 
control strategies, it can be concluded that long-term measures 
may not be as effective, as the environmental conditions present 
at the beginning of this analysis would no longer be the same when 
the measure is applied.
\\\\
As future work, it would be possible to compare the general 
methodology used by the referenced authors with the methodology 
used in this work and identify what could be modified so that, 
using the original parameters, the model still reaches endemic equilibrium.
\\\\
Furthermore, as verified in the equilibrium calculations, 
something else that could be studied as a continuation of 
the work is the analysis of its evolution over time since 
they are given by oscillatory functions, such as the biting 
rate, birth rate, mortality rate, survival probability, and 
the infection rate of exposed individuals, which vary 
depending on temperature and precipitation factors, as presented previously. 
\\\\
Another possible option would be to analyze the use of $k$ in relation to other 
parameters instead of only the infection rate,
such as the mosquito's birth and mortality rates, daily survival rates, among others.
