\chapter{Conclusion}

Throughout the development of the undergraduate thesis, 
different modifications to the dynamics of malaria transmission in the Amazon were explored 
in order to align the modeling more closely with the natural history of the disease in this 
environment. The ultimate goal was to understand how ecological impacts in the 
region affect interactions between the vector and the host.
\\\\
With the obtained results, it was possible to 
perceive the effect that increased contact between humans and mosquitoes due 
to deforestation can have on malaria dynamics, based on the proportion of bites 
causing infection. Furthermore, it was observed how, depending on the original 
parameters provided, a much higher proximity between vector and host would be 
required for the disease to become endemic in the Amazon region. As verified, 
this contact could be double the normal amount, and yet it is still insufficient 
for the disease to become an epidemic.
\\\\
To further align the methods used with observed behaviors in reality, 
the application of a stochastic transmission model, incorporating constantly 
changing environmental variables, could be ideal. However, for the proposed 
purpose, the deterministic model used was sufficient to highlight the disease's 
sensitivity to climate and environmental changes, allowing a clear and focused 
analysis of interactions between vector and host. This provides a solid foundation 
for investigating the implications of environmental changes on malaria 
transmission and for future research and improvements in the model.
